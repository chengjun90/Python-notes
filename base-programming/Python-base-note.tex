\documentclass[12pt]{ctexart}

    
    
    \usepackage[T1]{fontenc}
    % Nicer default font (+ math font) than Computer Modern for most use cases
    \usepackage{mathpazo}

    % Basic figure setup, for now with no caption control since it's done
    % automatically by Pandoc (which extracts ![](path) syntax from Markdown).
    \usepackage{graphicx}
    % We will generate all images so they have a width \maxwidth. This means
    % that they will get their normal width if they fit onto the page, but
    % are scaled down if they would overflow the margins.
    \makeatletter
    \def\maxwidth{\ifdim\Gin@nat@width>\linewidth\linewidth
    \else\Gin@nat@width\fi}
    \makeatother
    \let\Oldincludegraphics\includegraphics
    % Set max figure width to be 80% of text width, for now hardcoded.
    \renewcommand{\includegraphics}[1]{\Oldincludegraphics[width=.8\maxwidth]{#1}}
    % Ensure that by default, figures have no caption (until we provide a
    % proper Figure object with a Caption API and a way to capture that
    % in the conversion process - todo).
    \usepackage{caption}
    \DeclareCaptionLabelFormat{nolabel}{}
    \captionsetup{labelformat=nolabel}

    \usepackage{adjustbox} % Used to constrain images to a maximum size 
    \usepackage{xcolor} % Allow colors to be defined
    \usepackage{enumerate} % Needed for markdown enumerations to work
    \usepackage{geometry} % Used to adjust the document margins
    \usepackage{amsmath} % Equations
    \usepackage{amssymb} % Equations
    \usepackage{textcomp} % defines textquotesingle
    % Hack from http://tex.stackexchange.com/a/47451/13684:
    \AtBeginDocument{%
        \def\PYZsq{\textquotesingle}% Upright quotes in Pygmentized code
    }
    \usepackage{upquote} % Upright quotes for verbatim code
    \usepackage{eurosym} % defines \euro
    \usepackage[mathletters]{ucs} % Extended unicode (utf-8) support
    \usepackage[utf8x]{inputenc} % Allow utf-8 characters in the tex document
    \usepackage{fancyvrb} % verbatim replacement that allows latex
    \usepackage{grffile} % extends the file name processing of package graphics 
                         % to support a larger range 
    % The hyperref package gives us a pdf with properly built
    % internal navigation ('pdf bookmarks' for the table of contents,
    % internal cross-reference links, web links for URLs, etc.)
    \usepackage{hyperref}
    \usepackage{longtable} % longtable support required by pandoc >1.10
    \usepackage{booktabs}  % table support for pandoc > 1.12.2
    \usepackage[inline]{enumitem} % IRkernel/repr support (it uses the enumerate* environment)
    \usepackage[normalem]{ulem} % ulem is needed to support strikethroughs (\sout)
                                % normalem makes italics be italics, not underlines
    \usepackage{mathrsfs}
    

    
    
    % Colors for the hyperref package
    \definecolor{urlcolor}{rgb}{0,.145,.698}
    \definecolor{linkcolor}{rgb}{.71,0.21,0.01}
    \definecolor{citecolor}{rgb}{.12,.54,.11}

    % ANSI colors
    \definecolor{ansi-black}{HTML}{3E424D}
    \definecolor{ansi-black-intense}{HTML}{282C36}
    \definecolor{ansi-red}{HTML}{E75C58}
    \definecolor{ansi-red-intense}{HTML}{B22B31}
    \definecolor{ansi-green}{HTML}{00A250}
    \definecolor{ansi-green-intense}{HTML}{007427}
    \definecolor{ansi-yellow}{HTML}{DDB62B}
    \definecolor{ansi-yellow-intense}{HTML}{B27D12}
    \definecolor{ansi-blue}{HTML}{208FFB}
    \definecolor{ansi-blue-intense}{HTML}{0065CA}
    \definecolor{ansi-magenta}{HTML}{D160C4}
    \definecolor{ansi-magenta-intense}{HTML}{A03196}
    \definecolor{ansi-cyan}{HTML}{60C6C8}
    \definecolor{ansi-cyan-intense}{HTML}{258F8F}
    \definecolor{ansi-white}{HTML}{C5C1B4}
    \definecolor{ansi-white-intense}{HTML}{A1A6B2}
    \definecolor{ansi-default-inverse-fg}{HTML}{FFFFFF}
    \definecolor{ansi-default-inverse-bg}{HTML}{000000}

    % commands and environments needed by pandoc snippets
    % extracted from the output of `pandoc -s`
    \providecommand{\tightlist}{%
      \setlength{\itemsep}{0pt}\setlength{\parskip}{0pt}}
    \DefineVerbatimEnvironment{Highlighting}{Verbatim}{commandchars=\\\{\}}
    % Add ',fontsize=\small' for more characters per line
    \newenvironment{Shaded}{}{}
    \newcommand{\KeywordTok}[1]{\textcolor[rgb]{0.00,0.44,0.13}{\textbf{{#1}}}}
    \newcommand{\DataTypeTok}[1]{\textcolor[rgb]{0.56,0.13,0.00}{{#1}}}
    \newcommand{\DecValTok}[1]{\textcolor[rgb]{0.25,0.63,0.44}{{#1}}}
    \newcommand{\BaseNTok}[1]{\textcolor[rgb]{0.25,0.63,0.44}{{#1}}}
    \newcommand{\FloatTok}[1]{\textcolor[rgb]{0.25,0.63,0.44}{{#1}}}
    \newcommand{\CharTok}[1]{\textcolor[rgb]{0.25,0.44,0.63}{{#1}}}
    \newcommand{\StringTok}[1]{\textcolor[rgb]{0.25,0.44,0.63}{{#1}}}
    \newcommand{\CommentTok}[1]{\textcolor[rgb]{0.38,0.63,0.69}{\textit{{#1}}}}
    \newcommand{\OtherTok}[1]{\textcolor[rgb]{0.00,0.44,0.13}{{#1}}}
    \newcommand{\AlertTok}[1]{\textcolor[rgb]{1.00,0.00,0.00}{\textbf{{#1}}}}
    \newcommand{\FunctionTok}[1]{\textcolor[rgb]{0.02,0.16,0.49}{{#1}}}
    \newcommand{\RegionMarkerTok}[1]{{#1}}
    \newcommand{\ErrorTok}[1]{\textcolor[rgb]{1.00,0.00,0.00}{\textbf{{#1}}}}
    \newcommand{\NormalTok}[1]{{#1}}
    
    % Additional commands for more recent versions of Pandoc
    \newcommand{\ConstantTok}[1]{\textcolor[rgb]{0.53,0.00,0.00}{{#1}}}
    \newcommand{\SpecialCharTok}[1]{\textcolor[rgb]{0.25,0.44,0.63}{{#1}}}
    \newcommand{\VerbatimStringTok}[1]{\textcolor[rgb]{0.25,0.44,0.63}{{#1}}}
    \newcommand{\SpecialStringTok}[1]{\textcolor[rgb]{0.73,0.40,0.53}{{#1}}}
    \newcommand{\ImportTok}[1]{{#1}}
    \newcommand{\DocumentationTok}[1]{\textcolor[rgb]{0.73,0.13,0.13}{\textit{{#1}}}}
    \newcommand{\AnnotationTok}[1]{\textcolor[rgb]{0.38,0.63,0.69}{\textbf{\textit{{#1}}}}}
    \newcommand{\CommentVarTok}[1]{\textcolor[rgb]{0.38,0.63,0.69}{\textbf{\textit{{#1}}}}}
    \newcommand{\VariableTok}[1]{\textcolor[rgb]{0.10,0.09,0.49}{{#1}}}
    \newcommand{\ControlFlowTok}[1]{\textcolor[rgb]{0.00,0.44,0.13}{\textbf{{#1}}}}
    \newcommand{\OperatorTok}[1]{\textcolor[rgb]{0.40,0.40,0.40}{{#1}}}
    \newcommand{\BuiltInTok}[1]{{#1}}
    \newcommand{\ExtensionTok}[1]{{#1}}
    \newcommand{\PreprocessorTok}[1]{\textcolor[rgb]{0.74,0.48,0.00}{{#1}}}
    \newcommand{\AttributeTok}[1]{\textcolor[rgb]{0.49,0.56,0.16}{{#1}}}
    \newcommand{\InformationTok}[1]{\textcolor[rgb]{0.38,0.63,0.69}{\textbf{\textit{{#1}}}}}
    \newcommand{\WarningTok}[1]{\textcolor[rgb]{0.38,0.63,0.69}{\textbf{\textit{{#1}}}}}
    
    
    % Define a nice break command that doesn't care if a line doesn't already
    % exist.
    \def\br{\hspace*{\fill} \\* }
    % Math Jax compatibility definitions
    \def\gt{>}
    \def\lt{<}
    \let\Oldtex\TeX
    \let\Oldlatex\LaTeX
    \renewcommand{\TeX}{\textrm{\Oldtex}}
    \renewcommand{\LaTeX}{\textrm{\Oldlatex}}
    % Document parameters
    % Document title
    \title{\textbf{\LARGE{Python编程基础学习笔记}}}
    \author{Cheng Jun}
    \date{\today}
    
    
\usepackage{xeCJK}
\setCJKmainfont{思源宋体 CN}[BoldFont = 思源宋体 CN Bold]
\setCJKsansfont{思源黑体 CN}
\setmainfont{思源宋体 CN}[BoldFont = 思源宋体 CN Bold]
\setsansfont{思源黑体 CN}

    % Pygments definitions
    
\makeatletter
\def\PY@reset{\let\PY@it=\relax \let\PY@bf=\relax%
    \let\PY@ul=\relax \let\PY@tc=\relax%
    \let\PY@bc=\relax \let\PY@ff=\relax}
\def\PY@tok#1{\csname PY@tok@#1\endcsname}
\def\PY@toks#1+{\ifx\relax#1\empty\else%
    \PY@tok{#1}\expandafter\PY@toks\fi}
\def\PY@do#1{\PY@bc{\PY@tc{\PY@ul{%
    \PY@it{\PY@bf{\PY@ff{#1}}}}}}}
\def\PY#1#2{\PY@reset\PY@toks#1+\relax+\PY@do{#2}}

\expandafter\def\csname PY@tok@w\endcsname{\def\PY@tc##1{\textcolor[rgb]{0.73,0.73,0.73}{##1}}}
\expandafter\def\csname PY@tok@c\endcsname{\let\PY@it=\textit\def\PY@tc##1{\textcolor[rgb]{0.25,0.50,0.50}{##1}}}
\expandafter\def\csname PY@tok@cp\endcsname{\def\PY@tc##1{\textcolor[rgb]{0.74,0.48,0.00}{##1}}}
\expandafter\def\csname PY@tok@k\endcsname{\let\PY@bf=\textbf\def\PY@tc##1{\textcolor[rgb]{0.00,0.50,0.00}{##1}}}
\expandafter\def\csname PY@tok@kp\endcsname{\def\PY@tc##1{\textcolor[rgb]{0.00,0.50,0.00}{##1}}}
\expandafter\def\csname PY@tok@kt\endcsname{\def\PY@tc##1{\textcolor[rgb]{0.69,0.00,0.25}{##1}}}
\expandafter\def\csname PY@tok@o\endcsname{\def\PY@tc##1{\textcolor[rgb]{0.40,0.40,0.40}{##1}}}
\expandafter\def\csname PY@tok@ow\endcsname{\let\PY@bf=\textbf\def\PY@tc##1{\textcolor[rgb]{0.67,0.13,1.00}{##1}}}
\expandafter\def\csname PY@tok@nb\endcsname{\def\PY@tc##1{\textcolor[rgb]{0.00,0.50,0.00}{##1}}}
\expandafter\def\csname PY@tok@nf\endcsname{\def\PY@tc##1{\textcolor[rgb]{0.00,0.00,1.00}{##1}}}
\expandafter\def\csname PY@tok@nc\endcsname{\let\PY@bf=\textbf\def\PY@tc##1{\textcolor[rgb]{0.00,0.00,1.00}{##1}}}
\expandafter\def\csname PY@tok@nn\endcsname{\let\PY@bf=\textbf\def\PY@tc##1{\textcolor[rgb]{0.00,0.00,1.00}{##1}}}
\expandafter\def\csname PY@tok@ne\endcsname{\let\PY@bf=\textbf\def\PY@tc##1{\textcolor[rgb]{0.82,0.25,0.23}{##1}}}
\expandafter\def\csname PY@tok@nv\endcsname{\def\PY@tc##1{\textcolor[rgb]{0.10,0.09,0.49}{##1}}}
\expandafter\def\csname PY@tok@no\endcsname{\def\PY@tc##1{\textcolor[rgb]{0.53,0.00,0.00}{##1}}}
\expandafter\def\csname PY@tok@nl\endcsname{\def\PY@tc##1{\textcolor[rgb]{0.63,0.63,0.00}{##1}}}
\expandafter\def\csname PY@tok@ni\endcsname{\let\PY@bf=\textbf\def\PY@tc##1{\textcolor[rgb]{0.60,0.60,0.60}{##1}}}
\expandafter\def\csname PY@tok@na\endcsname{\def\PY@tc##1{\textcolor[rgb]{0.49,0.56,0.16}{##1}}}
\expandafter\def\csname PY@tok@nt\endcsname{\let\PY@bf=\textbf\def\PY@tc##1{\textcolor[rgb]{0.00,0.50,0.00}{##1}}}
\expandafter\def\csname PY@tok@nd\endcsname{\def\PY@tc##1{\textcolor[rgb]{0.67,0.13,1.00}{##1}}}
\expandafter\def\csname PY@tok@s\endcsname{\def\PY@tc##1{\textcolor[rgb]{0.73,0.13,0.13}{##1}}}
\expandafter\def\csname PY@tok@sd\endcsname{\let\PY@it=\textit\def\PY@tc##1{\textcolor[rgb]{0.73,0.13,0.13}{##1}}}
\expandafter\def\csname PY@tok@si\endcsname{\let\PY@bf=\textbf\def\PY@tc##1{\textcolor[rgb]{0.73,0.40,0.53}{##1}}}
\expandafter\def\csname PY@tok@se\endcsname{\let\PY@bf=\textbf\def\PY@tc##1{\textcolor[rgb]{0.73,0.40,0.13}{##1}}}
\expandafter\def\csname PY@tok@sr\endcsname{\def\PY@tc##1{\textcolor[rgb]{0.73,0.40,0.53}{##1}}}
\expandafter\def\csname PY@tok@ss\endcsname{\def\PY@tc##1{\textcolor[rgb]{0.10,0.09,0.49}{##1}}}
\expandafter\def\csname PY@tok@sx\endcsname{\def\PY@tc##1{\textcolor[rgb]{0.00,0.50,0.00}{##1}}}
\expandafter\def\csname PY@tok@m\endcsname{\def\PY@tc##1{\textcolor[rgb]{0.40,0.40,0.40}{##1}}}
\expandafter\def\csname PY@tok@gh\endcsname{\let\PY@bf=\textbf\def\PY@tc##1{\textcolor[rgb]{0.00,0.00,0.50}{##1}}}
\expandafter\def\csname PY@tok@gu\endcsname{\let\PY@bf=\textbf\def\PY@tc##1{\textcolor[rgb]{0.50,0.00,0.50}{##1}}}
\expandafter\def\csname PY@tok@gd\endcsname{\def\PY@tc##1{\textcolor[rgb]{0.63,0.00,0.00}{##1}}}
\expandafter\def\csname PY@tok@gi\endcsname{\def\PY@tc##1{\textcolor[rgb]{0.00,0.63,0.00}{##1}}}
\expandafter\def\csname PY@tok@gr\endcsname{\def\PY@tc##1{\textcolor[rgb]{1.00,0.00,0.00}{##1}}}
\expandafter\def\csname PY@tok@ge\endcsname{\let\PY@it=\textit}
\expandafter\def\csname PY@tok@gs\endcsname{\let\PY@bf=\textbf}
\expandafter\def\csname PY@tok@gp\endcsname{\let\PY@bf=\textbf\def\PY@tc##1{\textcolor[rgb]{0.00,0.00,0.50}{##1}}}
\expandafter\def\csname PY@tok@go\endcsname{\def\PY@tc##1{\textcolor[rgb]{0.53,0.53,0.53}{##1}}}
\expandafter\def\csname PY@tok@gt\endcsname{\def\PY@tc##1{\textcolor[rgb]{0.00,0.27,0.87}{##1}}}
\expandafter\def\csname PY@tok@err\endcsname{\def\PY@bc##1{\setlength{\fboxsep}{0pt}\fcolorbox[rgb]{1.00,0.00,0.00}{1,1,1}{\strut ##1}}}
\expandafter\def\csname PY@tok@kc\endcsname{\let\PY@bf=\textbf\def\PY@tc##1{\textcolor[rgb]{0.00,0.50,0.00}{##1}}}
\expandafter\def\csname PY@tok@kd\endcsname{\let\PY@bf=\textbf\def\PY@tc##1{\textcolor[rgb]{0.00,0.50,0.00}{##1}}}
\expandafter\def\csname PY@tok@kn\endcsname{\let\PY@bf=\textbf\def\PY@tc##1{\textcolor[rgb]{0.00,0.50,0.00}{##1}}}
\expandafter\def\csname PY@tok@kr\endcsname{\let\PY@bf=\textbf\def\PY@tc##1{\textcolor[rgb]{0.00,0.50,0.00}{##1}}}
\expandafter\def\csname PY@tok@bp\endcsname{\def\PY@tc##1{\textcolor[rgb]{0.00,0.50,0.00}{##1}}}
\expandafter\def\csname PY@tok@fm\endcsname{\def\PY@tc##1{\textcolor[rgb]{0.00,0.00,1.00}{##1}}}
\expandafter\def\csname PY@tok@vc\endcsname{\def\PY@tc##1{\textcolor[rgb]{0.10,0.09,0.49}{##1}}}
\expandafter\def\csname PY@tok@vg\endcsname{\def\PY@tc##1{\textcolor[rgb]{0.10,0.09,0.49}{##1}}}
\expandafter\def\csname PY@tok@vi\endcsname{\def\PY@tc##1{\textcolor[rgb]{0.10,0.09,0.49}{##1}}}
\expandafter\def\csname PY@tok@vm\endcsname{\def\PY@tc##1{\textcolor[rgb]{0.10,0.09,0.49}{##1}}}
\expandafter\def\csname PY@tok@sa\endcsname{\def\PY@tc##1{\textcolor[rgb]{0.73,0.13,0.13}{##1}}}
\expandafter\def\csname PY@tok@sb\endcsname{\def\PY@tc##1{\textcolor[rgb]{0.73,0.13,0.13}{##1}}}
\expandafter\def\csname PY@tok@sc\endcsname{\def\PY@tc##1{\textcolor[rgb]{0.73,0.13,0.13}{##1}}}
\expandafter\def\csname PY@tok@dl\endcsname{\def\PY@tc##1{\textcolor[rgb]{0.73,0.13,0.13}{##1}}}
\expandafter\def\csname PY@tok@s2\endcsname{\def\PY@tc##1{\textcolor[rgb]{0.73,0.13,0.13}{##1}}}
\expandafter\def\csname PY@tok@sh\endcsname{\def\PY@tc##1{\textcolor[rgb]{0.73,0.13,0.13}{##1}}}
\expandafter\def\csname PY@tok@s1\endcsname{\def\PY@tc##1{\textcolor[rgb]{0.73,0.13,0.13}{##1}}}
\expandafter\def\csname PY@tok@mb\endcsname{\def\PY@tc##1{\textcolor[rgb]{0.40,0.40,0.40}{##1}}}
\expandafter\def\csname PY@tok@mf\endcsname{\def\PY@tc##1{\textcolor[rgb]{0.40,0.40,0.40}{##1}}}
\expandafter\def\csname PY@tok@mh\endcsname{\def\PY@tc##1{\textcolor[rgb]{0.40,0.40,0.40}{##1}}}
\expandafter\def\csname PY@tok@mi\endcsname{\def\PY@tc##1{\textcolor[rgb]{0.40,0.40,0.40}{##1}}}
\expandafter\def\csname PY@tok@il\endcsname{\def\PY@tc##1{\textcolor[rgb]{0.40,0.40,0.40}{##1}}}
\expandafter\def\csname PY@tok@mo\endcsname{\def\PY@tc##1{\textcolor[rgb]{0.40,0.40,0.40}{##1}}}
\expandafter\def\csname PY@tok@ch\endcsname{\let\PY@it=\textit\def\PY@tc##1{\textcolor[rgb]{0.25,0.50,0.50}{##1}}}
\expandafter\def\csname PY@tok@cm\endcsname{\let\PY@it=\textit\def\PY@tc##1{\textcolor[rgb]{0.25,0.50,0.50}{##1}}}
\expandafter\def\csname PY@tok@cpf\endcsname{\let\PY@it=\textit\def\PY@tc##1{\textcolor[rgb]{0.25,0.50,0.50}{##1}}}
\expandafter\def\csname PY@tok@c1\endcsname{\let\PY@it=\textit\def\PY@tc##1{\textcolor[rgb]{0.25,0.50,0.50}{##1}}}
\expandafter\def\csname PY@tok@cs\endcsname{\let\PY@it=\textit\def\PY@tc##1{\textcolor[rgb]{0.25,0.50,0.50}{##1}}}

\def\PYZbs{\char`\\}
\def\PYZus{\char`\_}
\def\PYZob{\char`\{}
\def\PYZcb{\char`\}}
\def\PYZca{\char`\^}
\def\PYZam{\char`\&}
\def\PYZlt{\char`\<}
\def\PYZgt{\char`\>}
\def\PYZsh{\char`\#}
\def\PYZpc{\char`\%}
\def\PYZdl{\char`\$}
\def\PYZhy{\char`\-}
\def\PYZsq{\char`\'}
\def\PYZdq{\char`\"}
\def\PYZti{\char`\~}
% for compatibility with earlier versions
\def\PYZat{@}
\def\PYZlb{[}
\def\PYZrb{]}
\makeatother


    % Exact colors from NB
    \definecolor{incolor}{rgb}{0.0, 0.0, 0.5}
    \definecolor{outcolor}{rgb}{0.545, 0.0, 0.0}



    
    % Prevent overflowing lines due to hard-to-break entities
    \sloppy 
    % Setup hyperref package
    \hypersetup{
      breaklinks=true,  % so long urls are correctly broken across lines
      colorlinks=true,
      urlcolor=urlcolor,
      linkcolor=linkcolor,
      citecolor=citecolor,
      pdfauthor={Cheng Jun},
            pdftitle={Python编程基础学习笔记},
            pdfsubject={编程与开发},
            pdfkeywords={Python,编程,笔记},
            pdfproducer={LaTeX},
            pdfcreator={XeLaTeX}
      }
    % Slightly bigger margins than the latex defaults
    
    \geometry{verbose,tmargin=1in,bmargin=1in,lmargin=1in,rmargin=1in}
    
    


    \begin{document}
    
    
    \maketitle
    \tableofcontents
    \clearpage
    
    

    
    \hypertarget{ux524dux8a00}{%
\section{前言}\label{ux524dux8a00}}

    \hypertarget{ux7b14ux8bb0ux4f5cux8005}{%
\subsection{笔记作者}\label{ux7b14ux8bb0ux4f5cux8005}}

实证研究小⻘年,日常研究和关注经济、金融与会计等领域的问题,主要采用计量经济学和其他数据分析手法撰写学术论
文和研究报告。研究之余,泛读文史哲,关注自由与开源动态。在日常研究和工作中,喜欢选用自由和开源工具,喜欢使用最新的软件工具。

邮箱:cheng081@qq.com

Github:https://github.com/chengjun90

欢迎交流。

    \hypertarget{ux7b14ux8bb0ux76eeux7684}{%
\subsection{笔记目的}\label{ux7b14ux8bb0ux76eeux7684}}

本学习笔记主要为\sout{特定领域的数据科学}(实证研究)服务,力求\textbf{简明够用即可},用于构建知识体系,快速浏览全局。

更多的细节都是干中搜:

\begin{itemize}
\tightlist
\item
  直接谷歌、百度
\item
  查找Python官方手册
\end{itemize}

    \hypertarget{pythonux7b80ux4ecb}{%
\subsection{Python简介}\label{pythonux7b80ux4ecb}}

Python是非常简洁易懂的通用编程语言。Python适应于多个应用领域,在数据科学、机器学习和深度学习等领域得到广泛的应用。

Python代码简洁、易读,上手快。缺点之一是性能不如C/C++等语言。

    \hypertarget{ux7b14ux8bb0ux7684ux7cfbux7edfux73afux5883}{%
\subsection{笔记的系统环境}\label{ux7b14ux8bb0ux7684ux7cfbux7edfux73afux5883}}

    \begin{Verbatim}[commandchars=\\\{\}]
{\color{incolor}In [{\color{incolor}1}]:} \PY{k+kn}{import} \PY{n+nn}{sys}
        \PY{k+kn}{import} \PY{n+nn}{platform}
        \PY{n}{f}\PY{l+s+s1}{\PYZsq{}}\PY{l+s+s1}{操作系统:}\PY{l+s+s1}{\PYZob{}}\PY{l+s+s1}{platform.platform()\PYZcb{};Python版本:}\PY{l+s+si}{\PYZob{}sys.version[0:5]\PYZcb{}}\PY{l+s+s1}{\PYZsq{}}
\end{Verbatim}

\begin{Verbatim}[commandchars=\\\{\}]
{\color{outcolor}Out[{\color{outcolor}1}]:} '操作系统:Windows-10-10.0.17134-SP0;Python版本:3.7.0'
\end{Verbatim}
            
    \begin{Verbatim}[commandchars=\\\{\}]
{\color{incolor}In [{\color{incolor}2}]:} \PY{k+kn}{from} \PY{n+nn}{IPython}\PY{n+nn}{.}\PY{n+nn}{core}\PY{n+nn}{.}\PY{n+nn}{interactiveshell} \PY{k}{import} \PY{n}{InteractiveShell}
        \PY{n}{InteractiveShell}\PY{o}{.}\PY{n}{ast\PYZus{}node\PYZus{}interactivity} \PY{o}{=} \PY{l+s+s2}{\PYZdq{}}\PY{l+s+s2}{all}\PY{l+s+s2}{\PYZdq{}}
\end{Verbatim}

    \hypertarget{ux53d8ux91cfux4e0eux6570ux636eux7c7bux578b}{%
\section{变量与数据类型}\label{ux53d8ux91cfux4e0eux6570ux636eux7c7bux578b}}

    \hypertarget{ux53d8ux91cf}{%
\subsection{变量}\label{ux53d8ux91cf}}

Python是动态类型,赋值时无需声明类型。
变量可以看作名称(id)与值的关联。

    \hypertarget{ux6280ux5de7}{%
\subsubsection{技巧}\label{ux6280ux5de7}}

    解包

    \begin{Verbatim}[commandchars=\\\{\}]
{\color{incolor}In [{\color{incolor}3}]:} \PY{n}{a}\PY{p}{,}\PY{n}{b}\PY{o}{=}\PY{l+s+s1}{\PYZsq{}}\PY{l+s+s1}{hi}\PY{l+s+s1}{\PYZsq{}}
        \PY{n}{a}
        \PY{n}{b}
        \PY{n}{a}\PY{p}{,}\PY{n}{b}\PY{o}{=}\PY{p}{[}\PY{l+s+s2}{\PYZdq{}}\PY{l+s+s2}{哈哈}\PY{l+s+s2}{\PYZdq{}}\PY{p}{,}\PY{l+s+s2}{\PYZdq{}}\PY{l+s+s2}{呵呵}\PY{l+s+s2}{\PYZdq{}}\PY{p}{]}
        \PY{n}{a}
        \PY{n}{b}
\end{Verbatim}

\begin{Verbatim}[commandchars=\\\{\}]
{\color{outcolor}Out[{\color{outcolor}3}]:} 'h'
\end{Verbatim}
            
\begin{Verbatim}[commandchars=\\\{\}]
{\color{outcolor}Out[{\color{outcolor}3}]:} 'i'
\end{Verbatim}
            
\begin{Verbatim}[commandchars=\\\{\}]
{\color{outcolor}Out[{\color{outcolor}3}]:} '哈哈'
\end{Verbatim}
            
\begin{Verbatim}[commandchars=\\\{\}]
{\color{outcolor}Out[{\color{outcolor}3}]:} '呵呵'
\end{Verbatim}
            
    多个变量赋值

    \begin{Verbatim}[commandchars=\\\{\}]
{\color{incolor}In [{\color{incolor}4}]:} \PY{n}{a}\PY{p}{,}\PY{n}{b}\PY{p}{,}\PY{n}{c}\PY{o}{=}\PY{l+s+s2}{\PYZdq{}}\PY{l+s+s2}{呵呵}\PY{l+s+s2}{\PYZdq{}}\PY{p}{,}\PY{l+s+s2}{\PYZdq{}}\PY{l+s+s2}{喝喝}\PY{l+s+s2}{\PYZdq{}}\PY{p}{,}\PY{l+s+s2}{\PYZdq{}}\PY{l+s+s2}{哈哈}\PY{l+s+s2}{\PYZdq{}}
        \PY{n}{a}
        \PY{n}{b}
        \PY{n}{c}
\end{Verbatim}

\begin{Verbatim}[commandchars=\\\{\}]
{\color{outcolor}Out[{\color{outcolor}4}]:} '呵呵'
\end{Verbatim}
            
\begin{Verbatim}[commandchars=\\\{\}]
{\color{outcolor}Out[{\color{outcolor}4}]:} '喝喝'
\end{Verbatim}
            
\begin{Verbatim}[commandchars=\\\{\}]
{\color{outcolor}Out[{\color{outcolor}4}]:} '哈哈'
\end{Verbatim}
            
    交换值

    \begin{Verbatim}[commandchars=\\\{\}]
{\color{incolor}In [{\color{incolor}5}]:} \PY{n}{a}\PY{o}{=}\PY{l+m+mi}{1}
        \PY{n}{b}\PY{o}{=}\PY{l+m+mi}{2}
        \PY{n}{a}\PY{p}{,}\PY{n}{b}\PY{o}{=}\PY{n}{b}\PY{p}{,}\PY{n}{a}
        \PY{n}{a}
        \PY{n}{b}
\end{Verbatim}

\begin{Verbatim}[commandchars=\\\{\}]
{\color{outcolor}Out[{\color{outcolor}5}]:} 2
\end{Verbatim}
            
\begin{Verbatim}[commandchars=\\\{\}]
{\color{outcolor}Out[{\color{outcolor}5}]:} 1
\end{Verbatim}
            
    \hypertarget{ux57faux672cux6570ux636eux7c7bux578b}{%
\subsection{基本数据类型}\label{ux57faux672cux6570ux636eux7c7bux578b}}

    \hypertarget{ux6570ux503c}{%
\subsubsection{数值}\label{ux6570ux503c}}

整型int和浮点型float。

高精度使用decimal模块。

Python内置对复数的计算,对复数处理的数学函数在模块cmath中。

    \begin{Verbatim}[commandchars=\\\{\}]
{\color{incolor}In [{\color{incolor}6}]:} \PY{n+nb}{type}\PY{p}{(}\PY{l+m+mi}{1}\PY{p}{)}
        \PY{n+nb}{type}\PY{p}{(}\PY{l+m+mf}{1.1}\PY{p}{)}
        \PY{n+nb}{type}\PY{p}{(}\PY{l+m+mi}{1}\PY{o}{+}\PY{l+m+mi}{2}\PY{n}{j}\PY{p}{)}
\end{Verbatim}

\begin{Verbatim}[commandchars=\\\{\}]
{\color{outcolor}Out[{\color{outcolor}6}]:} int
\end{Verbatim}
            
\begin{Verbatim}[commandchars=\\\{\}]
{\color{outcolor}Out[{\color{outcolor}6}]:} float
\end{Verbatim}
            
\begin{Verbatim}[commandchars=\\\{\}]
{\color{outcolor}Out[{\color{outcolor}6}]:} complex
\end{Verbatim}
            
    \begin{Verbatim}[commandchars=\\\{\}]
{\color{incolor}In [{\color{incolor}7}]:} \PY{p}{(}\PY{l+m+mi}{1}\PY{o}{+}\PY{l+m+mi}{2}\PY{n}{j}\PY{p}{)}\PY{o}{*}\PY{p}{(}\PY{l+m+mi}{2}\PY{o}{+}\PY{l+m+mi}{3}\PY{n}{j}\PY{p}{)}
\end{Verbatim}

\begin{Verbatim}[commandchars=\\\{\}]
{\color{outcolor}Out[{\color{outcolor}7}]:} (-4+7j)
\end{Verbatim}
            
    \hypertarget{ux5b57ux7b26}{%
\subsubsection{字符}\label{ux5b57ux7b26}}

Python中字符串不可修改,只能生成新的字符串。

字符串可以使用单引号、双引号来标记。字符串是多行的时候使用三引号。

    字符串格式化,直接使用f方法好了。

    \begin{Verbatim}[commandchars=\\\{\}]
{\color{incolor}In [{\color{incolor}8}]:} \PY{c+c1}{\PYZsh{} printf style}
        \PY{l+s+s1}{\PYZsq{}}\PY{l+s+s1}{我喜欢}\PY{l+s+si}{\PYZpc{}s}\PY{l+s+s1}{, 学习了}\PY{l+s+si}{\PYZpc{}d}\PY{l+s+s1}{年}\PY{l+s+s1}{\PYZsq{}}\PY{o}{\PYZpc{}}\PY{p}{(}\PY{l+s+s1}{\PYZsq{}}\PY{l+s+s1}{python}\PY{l+s+s1}{\PYZsq{}}\PY{p}{,} \PY{l+m+mi}{3}\PY{p}{)}
\end{Verbatim}

\begin{Verbatim}[commandchars=\\\{\}]
{\color{outcolor}Out[{\color{outcolor}8}]:} '我喜欢python, 学习了3年'
\end{Verbatim}
            
    \begin{Verbatim}[commandchars=\\\{\}]
{\color{incolor}In [{\color{incolor}9}]:} \PY{c+c1}{\PYZsh{} format方法}
        \PY{l+s+s1}{\PYZsq{}}\PY{l+s+s1}{我喜欢}\PY{l+s+si}{\PYZob{}\PYZcb{}}\PY{l+s+s1}{, 学习了}\PY{l+s+si}{\PYZob{}\PYZcb{}}\PY{l+s+s1}{年}\PY{l+s+s1}{\PYZsq{}}\PY{o}{.}\PY{n}{format}\PY{p}{(}\PY{l+s+s1}{\PYZsq{}}\PY{l+s+s1}{python}\PY{l+s+s1}{\PYZsq{}}\PY{p}{,}\PY{l+m+mi}{3}\PY{p}{)}
        \PY{l+s+s1}{\PYZsq{}}\PY{l+s+s1}{我喜欢}\PY{l+s+si}{\PYZob{}a\PYZcb{}}\PY{l+s+s1}{, 学习了}\PY{l+s+si}{\PYZob{}b\PYZcb{}}\PY{l+s+s1}{年}\PY{l+s+s1}{\PYZsq{}}\PY{o}{.}\PY{n}{format}\PY{p}{(}\PY{n}{b}\PY{o}{=}\PY{l+m+mi}{3}\PY{p}{,}\PY{n}{a}\PY{o}{=}\PY{l+s+s1}{\PYZsq{}}\PY{l+s+s1}{python}\PY{l+s+s1}{\PYZsq{}}\PY{p}{)}
\end{Verbatim}

\begin{Verbatim}[commandchars=\\\{\}]
{\color{outcolor}Out[{\color{outcolor}9}]:} '我喜欢python, 学习了3年'
\end{Verbatim}
            
\begin{Verbatim}[commandchars=\\\{\}]
{\color{outcolor}Out[{\color{outcolor}9}]:} '我喜欢python, 学习了3年'
\end{Verbatim}
            
    \begin{Verbatim}[commandchars=\\\{\}]
{\color{incolor}In [{\color{incolor}10}]:} \PY{c+c1}{\PYZsh{} f方法}
         \PY{n}{a}\PY{o}{=}\PY{l+s+s2}{\PYZdq{}}\PY{l+s+s2}{Python}\PY{l+s+s2}{\PYZdq{}}
         \PY{n}{b}\PY{o}{=}\PY{l+m+mi}{3}
         \PY{n}{f}\PY{l+s+s1}{\PYZsq{}}\PY{l+s+s1}{我喜欢}\PY{l+s+si}{\PYZob{}a\PYZcb{}}\PY{l+s+s1}{, 学习了}\PY{l+s+si}{\PYZob{}b\PYZcb{}}\PY{l+s+s1}{年}\PY{l+s+s1}{\PYZsq{}}
\end{Verbatim}

\begin{Verbatim}[commandchars=\\\{\}]
{\color{outcolor}Out[{\color{outcolor}10}]:} '我喜欢Python, 学习了3年'
\end{Verbatim}
            
    \hypertarget{ux903bux8f91ux503c}{%
\subsubsection{逻辑值}\label{ux903bux8f91ux503c}}

逻辑值仅包括True/False两个,用来配合if/while等语句做条件判断,其它数据类型可以转换为逻辑值。

标准的布尔运算符为and , or, not。

    \hypertarget{ux57faux672cux6570ux636eux7c7bux578bux8f6cux6362}{%
\subsection{基本数据类型转换}\label{ux57faux672cux6570ux636eux7c7bux578bux8f6cux6362}}

\begin{longtable}[]{@{}ll@{}}
\toprule
方法 & 说明\tabularnewline
\midrule
\endhead
int(x{[},base{]}) & 将x转换为一个整数\tabularnewline
float(x) & 将x转换到一个浮点数\tabularnewline
complex(real{[},imag{]}) & 创建一个复数\tabularnewline
str(x) & 将对象x转换为字符串\tabularnewline
repr(x) & 将对象x转换为表达式字符串\tabularnewline
eval(str) &
用来计算在字符串中的有效Python表达式,并返回一个对象\tabularnewline
tuple(s) & 将序列s转换为一个元组\tabularnewline
list(s) & 将序列s转换为一个列表\tabularnewline
chr(x) & 将一个整数转换为一个字符\tabularnewline
unichr(x) & 将一个整数转换为Unicode字符\tabularnewline
ord(x) & 将一个字符转换为它的整数值\tabularnewline
hex(x) & 将一个整数转换为一个十六进制字符串\tabularnewline
oct(x) & 将一个整数转换为一个八进制字符串\tabularnewline
\bottomrule
\end{longtable}

    \hypertarget{ux5b57ux8282ux548cunicode}{%
\subsection{字节和Unicode}\label{ux5b57ux8282ux548cunicode}}

在Python 3及以上版本中,字符串类型是Unicode编码。

可以用encode将这个Unicode字符串编码为UTF-8:

    \begin{Verbatim}[commandchars=\\\{\}]
{\color{incolor}In [{\color{incolor}11}]:} \PY{n}{var}\PY{o}{=}\PY{l+s+s2}{\PYZdq{}}\PY{l+s+s2}{呵呵}\PY{l+s+s2}{\PYZdq{}}
         \PY{n}{var}\PY{o}{.}\PY{n}{encode}\PY{p}{(}\PY{l+s+s1}{\PYZsq{}}\PY{l+s+s1}{utf\PYZhy{}8}\PY{l+s+s1}{\PYZsq{}}\PY{p}{)}
         \PY{n+nb}{type}\PY{p}{(}\PY{n}{var}\PY{o}{.}\PY{n}{encode}\PY{p}{(}\PY{l+s+s1}{\PYZsq{}}\PY{l+s+s1}{utf\PYZhy{}8}\PY{l+s+s1}{\PYZsq{}}\PY{p}{)}\PY{p}{)}
         \PY{n}{var}\PY{o}{.}\PY{n}{encode}\PY{p}{(}\PY{l+s+s1}{\PYZsq{}}\PY{l+s+s1}{utf\PYZhy{}8}\PY{l+s+s1}{\PYZsq{}}\PY{p}{)}\PY{o}{.}\PY{n}{decode}\PY{p}{(}\PY{l+s+s1}{\PYZsq{}}\PY{l+s+s1}{utf\PYZhy{}8}\PY{l+s+s1}{\PYZsq{}}\PY{p}{)}
\end{Verbatim}

\begin{Verbatim}[commandchars=\\\{\}]
{\color{outcolor}Out[{\color{outcolor}11}]:} b'\textbackslash{}xe5\textbackslash{}x91\textbackslash{}xb5\textbackslash{}xe5\textbackslash{}x91\textbackslash{}xb5'
\end{Verbatim}
            
\begin{Verbatim}[commandchars=\\\{\}]
{\color{outcolor}Out[{\color{outcolor}11}]:} bytes
\end{Verbatim}
            
\begin{Verbatim}[commandchars=\\\{\}]
{\color{outcolor}Out[{\color{outcolor}11}]:} '呵呵'
\end{Verbatim}
            
    \hypertarget{ux6570ux636eux7ed3ux6784}{%
\section{数据结构}\label{ux6570ux636eux7ed3ux6784}}

数据结构是用来存储一系列相关数据的集合。

数据结构也可以称之为容器。Python提供了许多高效的容器类型,其中可以存储对象的集合。

Python中有四种内置的数据结构:列表(List)、元组(Tuple)、字典(Dictionary)和集合(Set)。

\begin{itemize}
\tightlist
\item
  列表(list):列表则可以删除、 添加、 替换、
  重排序列中的元素。可变类型。列表是异质的,这意味着列表中的数据不必要同一个类。
\item
  元组(Tuple):元组是不可变类型。
\item
  字典(Dictionary):字典是通过键值key来索引元素value,而不是象列表是通过连续的整数来索引。字典是可变类型,可以添加、删除、替换元素。
\item
  集合:集合是不重复元素的无序组合。用set()创建空集,可用set()从其它序列转换生成集合。
\end{itemize}

    \begin{Verbatim}[commandchars=\\\{\}]
{\color{incolor}In [{\color{incolor}12}]:} \PY{n}{mylist}\PY{o}{=}\PY{p}{[}\PY{l+s+s1}{\PYZsq{}}\PY{l+s+s1}{a}\PY{l+s+s1}{\PYZsq{}}\PY{p}{,}\PY{l+s+s1}{\PYZsq{}}\PY{l+s+s1}{b}\PY{l+s+s1}{\PYZsq{}}\PY{p}{,}\PY{l+s+s1}{\PYZsq{}}\PY{l+s+s1}{b}\PY{l+s+s1}{\PYZsq{}}\PY{p}{,}\PY{l+m+mi}{123}\PY{p}{]}
         \PY{n}{mylist}
\end{Verbatim}

\begin{Verbatim}[commandchars=\\\{\}]
{\color{outcolor}Out[{\color{outcolor}12}]:} ['a', 'b', 'b', 123]
\end{Verbatim}
            
    \begin{Verbatim}[commandchars=\\\{\}]
{\color{incolor}In [{\color{incolor}13}]:} \PY{n}{mytuple}\PY{o}{=}\PY{p}{(}\PY{l+s+s1}{\PYZsq{}}\PY{l+s+s1}{a}\PY{l+s+s1}{\PYZsq{}}\PY{p}{,}\PY{l+s+s1}{\PYZsq{}}\PY{l+s+s1}{b}\PY{l+s+s1}{\PYZsq{}}\PY{p}{,}\PY{l+s+s1}{\PYZsq{}}\PY{l+s+s1}{b}\PY{l+s+s1}{\PYZsq{}}\PY{p}{,}\PY{l+m+mi}{123}\PY{p}{)}
         \PY{n}{mytuple}
\end{Verbatim}

\begin{Verbatim}[commandchars=\\\{\}]
{\color{outcolor}Out[{\color{outcolor}13}]:} ('a', 'b', 'b', 123)
\end{Verbatim}
            
    \begin{Verbatim}[commandchars=\\\{\}]
{\color{incolor}In [{\color{incolor}14}]:} \PY{n}{mydict}\PY{o}{=}\PY{p}{\PYZob{}}\PY{l+s+s2}{\PYZdq{}}\PY{l+s+s2}{name}\PY{l+s+s2}{\PYZdq{}}\PY{p}{:}\PY{l+s+s2}{\PYZdq{}}\PY{l+s+s2}{zhangsan}\PY{l+s+s2}{\PYZdq{}}\PY{p}{,}\PY{l+s+s2}{\PYZdq{}}\PY{l+s+s2}{age}\PY{l+s+s2}{\PYZdq{}}\PY{p}{:}\PY{l+m+mi}{18}\PY{p}{\PYZcb{}}
         \PY{n}{mydict}
\end{Verbatim}

\begin{Verbatim}[commandchars=\\\{\}]
{\color{outcolor}Out[{\color{outcolor}14}]:} \{'name': 'zhangsan', 'age': 18\}
\end{Verbatim}
            
    \begin{Verbatim}[commandchars=\\\{\}]
{\color{incolor}In [{\color{incolor}15}]:} \PY{n}{myset} \PY{o}{=} \PY{p}{\PYZob{}}\PY{l+m+mi}{1}\PY{p}{,}\PY{l+m+mi}{2}\PY{p}{,}\PY{l+m+mi}{1}\PY{p}{,}\PY{l+m+mi}{3}\PY{p}{\PYZcb{}}
         \PY{n}{myset}
\end{Verbatim}

\begin{Verbatim}[commandchars=\\\{\}]
{\color{outcolor}Out[{\color{outcolor}15}]:} \{1, 2, 3\}
\end{Verbatim}
            
    Python 3.7引入了dataclass 。

dataclass一个简单的数据容器,支持通过的对象属性进行访问。

    \begin{Verbatim}[commandchars=\\\{\}]
{\color{incolor}In [{\color{incolor}16}]:} \PY{k+kn}{from} \PY{n+nn}{dataclasses} \PY{k}{import} \PY{n}{dataclass}
         \PY{k+kn}{from} \PY{n+nn}{typing} \PY{k}{import} \PY{n}{Any}
         
         \PY{n+nd}{@dataclass}
         \PY{k}{class} \PY{n+nc}{Student}\PY{p}{:}
             \PY{n}{name}\PY{p}{:} \PY{n+nb}{str}
             \PY{n}{age}\PY{p}{:} \PY{n+nb}{int}\PY{o}{=}\PY{l+m+mi}{18}
             \PY{n}{score}\PY{p}{:} \PY{n}{Any}\PY{o}{=}\PY{l+s+s2}{\PYZdq{}}\PY{l+s+s2}{\PYZdq{}}
             
             \PY{k}{def} \PY{n+nf}{level}\PY{p}{(}\PY{n+nb+bp}{self}\PY{p}{,}\PY{n}{hour}\PY{p}{)}\PY{p}{:}
                 \PY{k}{if} \PY{n+nb+bp}{self}\PY{o}{.}\PY{n}{score}\PY{o}{/}\PY{n}{hour}\PY{o}{\PYZgt{}}\PY{l+m+mi}{1}\PY{p}{:}
                     \PY{k}{return} \PY{l+s+s2}{\PYZdq{}}\PY{l+s+s2}{优秀}\PY{l+s+s2}{\PYZdq{}}
                 \PY{k}{else}\PY{p}{:}
                     \PY{k}{return} \PY{l+s+s2}{\PYZdq{}}\PY{l+s+s2}{其他}\PY{l+s+s2}{\PYZdq{}}
                 
         \PY{n}{lisi} \PY{o}{=} \PY{n}{Student}\PY{p}{(}\PY{n}{name}\PY{o}{=}\PY{l+s+s2}{\PYZdq{}}\PY{l+s+s2}{李四}\PY{l+s+s2}{\PYZdq{}}\PY{p}{,}\PY{n}{score}\PY{o}{=}\PY{l+m+mf}{92.5}\PY{p}{)}
         \PY{n}{lisi}\PY{o}{.}\PY{n}{level}\PY{p}{(}\PY{l+m+mi}{20}\PY{p}{)}
\end{Verbatim}

\begin{Verbatim}[commandchars=\\\{\}]
{\color{outcolor}Out[{\color{outcolor}16}]:} '优秀'
\end{Verbatim}
            
    \hypertarget{ux5224ux65adux4e0eux5faaux73af}{%
\section{判断与循环}\label{ux5224ux65adux4e0eux5faaux73af}}

    \hypertarget{ux5224ux65ad}{%
\subsection{判断}\label{ux5224ux65ad}}

if-elif-else语句

    \begin{Verbatim}[commandchars=\\\{\}]
{\color{incolor}In [{\color{incolor}17}]:} \PY{k+kn}{import} \PY{n+nn}{random}
         \PY{n}{a}\PY{o}{=}\PY{n}{random}\PY{o}{.}\PY{n}{gauss}\PY{p}{(}\PY{l+m+mi}{0}\PY{p}{,}\PY{l+m+mi}{1}\PY{p}{)}
         
         \PY{k}{if} \PY{n}{a}\PY{o}{\PYZgt{}}\PY{o}{=}\PY{l+m+mf}{1.65}\PY{p}{:}
             \PY{n+nb}{print}\PY{p}{(}\PY{l+s+s2}{\PYZdq{}}\PY{l+s+s2}{真大}\PY{l+s+s2}{\PYZdq{}}\PY{p}{)}
         \PY{k}{elif} \PY{n}{a}\PY{o}{\PYZgt{}}\PY{l+m+mi}{0}\PY{p}{:}
             \PY{n+nb}{print}\PY{p}{(}\PY{l+s+s2}{\PYZdq{}}\PY{l+s+s2}{还行}\PY{l+s+s2}{\PYZdq{}}\PY{p}{)}
         \PY{k}{else}\PY{p}{:}
             \PY{n+nb}{print}\PY{p}{(}\PY{l+s+s2}{\PYZdq{}}\PY{l+s+s2}{结束}\PY{l+s+s2}{\PYZdq{}}\PY{p}{)}
\end{Verbatim}

    \begin{Verbatim}[commandchars=\\\{\}]
结束

    \end{Verbatim}

    三元操作符是 if-else 语句也就是条件操作符的一个快捷方式:

{[}表达式为真的返回值{]} if {[}表达式{]} else {[}表达式为假的返回值{]}

    \begin{Verbatim}[commandchars=\\\{\}]
{\color{incolor}In [{\color{incolor}18}]:} \PY{n}{a}\PY{o}{=}\PY{n}{random}\PY{o}{.}\PY{n}{gauss}\PY{p}{(}\PY{l+m+mi}{0}\PY{p}{,}\PY{l+m+mi}{1}\PY{p}{)}
         
         \PY{l+s+s2}{\PYZdq{}}\PY{l+s+s2}{还行}\PY{l+s+s2}{\PYZdq{}} \PY{k}{if} \PY{n}{a}\PY{o}{\PYZgt{}}\PY{o}{=}\PY{l+m+mi}{0} \PY{k}{else} \PY{l+s+s2}{\PYZdq{}}\PY{l+s+s2}{结束}\PY{l+s+s2}{\PYZdq{}}
\end{Verbatim}

\begin{Verbatim}[commandchars=\\\{\}]
{\color{outcolor}Out[{\color{outcolor}18}]:} '还行'
\end{Verbatim}
            
    \hypertarget{ux5faaux73af}{%
\subsection{循环}\label{ux5faaux73af}}

Python 提供了 for 循环和 while 循环。

迭代循环for:

\begin{Shaded}
\begin{Highlighting}[]
\ControlFlowTok{for} \OperatorTok{<}\NormalTok{变量}\OperatorTok{>} \KeywordTok{in} \OperatorTok{<}\NormalTok{可迭代对象}\OperatorTok{>}\NormalTok{:}
    \OperatorTok{<}\NormalTok{语句块}\OperatorTok{>}
    \ControlFlowTok{break} \CommentTok{#跳出循环}
    \ControlFlowTok{continue} \CommentTok{#略过余下循环语句}
\ControlFlowTok{else}\NormalTok{: }\CommentTok{#迭代完毕,则执行}
    \OperatorTok{<}\NormalTok{语句块}\OperatorTok{>}
\end{Highlighting}
\end{Shaded}

while:

\begin{Shaded}
\begin{Highlighting}[]
\ControlFlowTok{while}\NormalTok{ 条件:}
\NormalTok{    语句块}
    \ControlFlowTok{break} \CommentTok{# 跳出循环}
    \ControlFlowTok{continue} \CommentTok{#跳出余下的语句块}
\NormalTok{    语句块}
\ControlFlowTok{else}\NormalTok{: }\CommentTok{#条件不满足则跳出,执行以下语句块}
\NormalTok{    语句块}
\end{Highlighting}
\end{Shaded}

控制循环的语句:

\begin{longtable}[]{@{}ll@{}}
\toprule
循环控制语句 & 描述\tabularnewline
\midrule
\endhead
break & 在语句块执行过程中终止循环,并且跳出整个循环\tabularnewline
continue &
在语句块执行过程中终止当前循环,跳出该次循环,执行下一次循环\tabularnewline
pass & pass 是空语句,是为了保持程序结构的完整性\tabularnewline
\bottomrule
\end{longtable}

    \begin{Verbatim}[commandchars=\\\{\}]
{\color{incolor}In [{\color{incolor}19}]:} \PY{n}{s}\PY{o}{=}\PY{l+s+s2}{\PYZdq{}}\PY{l+s+s2}{我的数据分析工具箱}\PY{l+s+s2}{\PYZdq{}}
         \PY{k}{for}  \PY{n}{i} \PY{o+ow}{in} \PY{l+s+s2}{\PYZdq{}}\PY{l+s+s2}{的箱}\PY{l+s+s2}{\PYZdq{}}\PY{p}{:}
             \PY{k}{if} \PY{n}{i} \PY{o+ow}{in} \PY{n}{s}\PY{p}{:}
                 \PY{n+nb}{print}\PY{p}{(}\PY{n}{i}\PY{p}{)}
\end{Verbatim}

    \begin{Verbatim}[commandchars=\\\{\}]
的
箱

    \end{Verbatim}

    \begin{Verbatim}[commandchars=\\\{\}]
{\color{incolor}In [{\color{incolor}20}]:} \PY{n}{list1} \PY{o}{=} \PY{p}{[}\PY{l+s+s2}{\PYZdq{}}\PY{l+s+s2}{这是}\PY{l+s+s2}{\PYZdq{}}\PY{p}{,} \PY{l+s+s2}{\PYZdq{}}\PY{l+s+s2}{一个}\PY{l+s+s2}{\PYZdq{}}\PY{p}{,} \PY{l+s+s2}{\PYZdq{}}\PY{l+s+s2}{测试}\PY{l+s+s2}{\PYZdq{}}\PY{p}{]}
         
         \PY{k}{for} \PY{n}{index}\PY{p}{,} \PY{n}{item} \PY{o+ow}{in} \PY{n+nb}{enumerate}\PY{p}{(}\PY{n}{list1}\PY{p}{)}\PY{p}{:}
             \PY{n+nb}{print}\PY{p}{(}\PY{n}{index}\PY{p}{,} \PY{n}{item}\PY{p}{)}
         
         \PY{k}{for} \PY{n}{index}\PY{p}{,} \PY{n}{item} \PY{o+ow}{in} \PY{n+nb}{enumerate}\PY{p}{(}\PY{n}{list1}\PY{p}{,} \PY{l+m+mi}{100}\PY{p}{)}\PY{p}{:}
             \PY{n+nb}{print}\PY{p}{(}\PY{n}{index}\PY{p}{,} \PY{n}{item}\PY{p}{)}
\end{Verbatim}

    \begin{Verbatim}[commandchars=\\\{\}]
0 这是
1 一个
2 测试
100 这是
101 一个
102 测试

    \end{Verbatim}

    \begin{Verbatim}[commandchars=\\\{\}]
{\color{incolor}In [{\color{incolor}21}]:} \PY{n}{a}\PY{o}{=}\PY{p}{[}\PY{l+m+mi}{1}\PY{p}{,}\PY{l+m+mi}{2}\PY{p}{,}\PY{l+m+mi}{3}\PY{p}{]}
         \PY{n}{b}\PY{o}{=}\PY{p}{[}\PY{l+s+s2}{\PYZdq{}}\PY{l+s+s2}{a}\PY{l+s+s2}{\PYZdq{}}\PY{p}{,}\PY{l+s+s2}{\PYZdq{}}\PY{l+s+s2}{b}\PY{l+s+s2}{\PYZdq{}}\PY{p}{,}\PY{l+s+s2}{\PYZdq{}}\PY{l+s+s2}{c}\PY{l+s+s2}{\PYZdq{}}\PY{p}{]}
         \PY{n}{c}\PY{o}{=}\PY{n+nb}{zip}\PY{p}{(}\PY{n}{a}\PY{p}{,}\PY{n}{b}\PY{p}{)}
         \PY{n}{c}
         \PY{n+nb}{list}\PY{p}{(}\PY{n}{c}\PY{p}{)}
         
         \PY{k}{for} \PY{n}{i}\PY{p}{,}\PY{n}{j} \PY{o+ow}{in} \PY{n+nb}{zip}\PY{p}{(}\PY{n}{a}\PY{p}{,}\PY{n}{b}\PY{p}{)}\PY{p}{:}
             \PY{n+nb}{print}\PY{p}{(}\PY{n}{i}\PY{p}{,}\PY{n}{j}\PY{p}{)}
\end{Verbatim}

\begin{Verbatim}[commandchars=\\\{\}]
{\color{outcolor}Out[{\color{outcolor}21}]:} <zip at 0x1903cffcc88>
\end{Verbatim}
            
\begin{Verbatim}[commandchars=\\\{\}]
{\color{outcolor}Out[{\color{outcolor}21}]:} [(1, 'a'), (2, 'b'), (3, 'c')]
\end{Verbatim}
            
    \begin{Verbatim}[commandchars=\\\{\}]
1 a
2 b
3 c

    \end{Verbatim}

    \begin{Verbatim}[commandchars=\\\{\}]
{\color{incolor}In [{\color{incolor}22}]:} \PY{n}{a}\PY{o}{=}\PY{l+m+mi}{4}
         
         \PY{k}{while} \PY{n}{a}\PY{p}{:}
             \PY{n+nb}{print}\PY{p}{(}\PY{n}{a}\PY{p}{)}
             \PY{k}{if} \PY{n}{a}\PY{o}{\PYZlt{}}\PY{l+m+mi}{3}\PY{p}{:}
                 \PY{k}{break}
             \PY{n}{a}\PY{o}{=}\PY{n}{a}\PY{o}{\PYZhy{}}\PY{l+m+mi}{1}
         \PY{k}{else}\PY{p}{:}
             \PY{n+nb}{print}\PY{p}{(}\PY{l+s+s2}{\PYZdq{}}\PY{l+s+s2}{结束}\PY{l+s+s2}{\PYZdq{}}\PY{p}{)}
\end{Verbatim}

    \begin{Verbatim}[commandchars=\\\{\}]
4
3
2

    \end{Verbatim}

    \hypertarget{ux8fd0ux7b97ux7b26}{%
\subsection{运算符:}\label{ux8fd0ux7b97ux7b26}}

Python语言支持以下类型运算符:

\begin{itemize}
\tightlist
\item
  算术运算符
\item
  比较运算符
\item
  赋值运算符
\item
  逻辑运算符:not,and,or
\item
  成员操作符:in,not in
\item
  身份运算符:is
\end{itemize}

    \hypertarget{ux9ad8ux7ea7ux7279ux6027}{%
\section{高级特性}\label{ux9ad8ux7ea7ux7279ux6027}}

    \hypertarget{ux5207ux7247}{%
\subsection{切片}\label{ux5207ux7247}}

取一个list或tuple的部分元素是非常常见的操作。比如,一个list如下:

    \begin{Verbatim}[commandchars=\\\{\}]
{\color{incolor}In [{\color{incolor}23}]:} \PY{n}{L} \PY{o}{=} \PY{p}{[}\PY{l+s+s1}{\PYZsq{}}\PY{l+s+s1}{a}\PY{l+s+s1}{\PYZsq{}}\PY{p}{,} \PY{l+s+s1}{\PYZsq{}}\PY{l+s+s1}{b}\PY{l+s+s1}{\PYZsq{}}\PY{p}{,} \PY{l+s+s1}{\PYZsq{}}\PY{l+s+s1}{c}\PY{l+s+s1}{\PYZsq{}}\PY{p}{,} \PY{l+s+s1}{\PYZsq{}}\PY{l+s+s1}{d}\PY{l+s+s1}{\PYZsq{}}\PY{p}{,} \PY{l+s+s1}{\PYZsq{}}\PY{l+s+s1}{e}\PY{l+s+s1}{\PYZsq{}}\PY{p}{]}
         \PY{n}{L}\PY{p}{[}\PY{l+m+mi}{2}\PY{p}{:}\PY{l+m+mi}{4}\PY{p}{]}
         
         
         \PY{n}{strs}\PY{o}{=}\PY{l+s+s2}{\PYZdq{}}\PY{l+s+s2}{取一个list或tuple的部分元素是非常常见的操作。}\PY{l+s+s2}{\PYZdq{}}
         \PY{n}{strs}\PY{p}{[}\PY{o}{\PYZhy{}}\PY{l+m+mi}{6}\PY{p}{:}\PY{p}{:}\PY{p}{]}
\end{Verbatim}

\begin{Verbatim}[commandchars=\\\{\}]
{\color{outcolor}Out[{\color{outcolor}23}]:} ['c', 'd']
\end{Verbatim}
            
\begin{Verbatim}[commandchars=\\\{\}]
{\color{outcolor}Out[{\color{outcolor}23}]:} '常见的操作。'
\end{Verbatim}
            
    \hypertarget{ux8fedux4ee3ux5668}{%
\subsection{迭代器}\label{ux8fedux4ee3ux5668}}

    \hypertarget{ux751fux6210ux5668}{%
\subsection{生成器}\label{ux751fux6210ux5668}}

可以节约内存。

    \begin{Verbatim}[commandchars=\\\{\}]
{\color{incolor}In [{\color{incolor}24}]:} \PY{n}{a}\PY{o}{=}\PY{p}{[}\PY{l+m+mi}{0}\PY{p}{,}\PY{l+m+mi}{1}\PY{p}{,}\PY{l+m+mi}{2}\PY{p}{,}\PY{l+m+mi}{3}\PY{p}{,}\PY{l+m+mi}{4}\PY{p}{,}\PY{l+m+mi}{5}\PY{p}{]}
         \PY{n}{a2}\PY{o}{=}\PY{p}{(}\PY{l+m+mi}{2}\PY{o}{*}\PY{n}{i} \PY{k}{for} \PY{n}{i} \PY{o+ow}{in} \PY{n}{a}\PY{p}{)}
         \PY{n}{a2}
         \PY{k}{for} \PY{n}{i} \PY{o+ow}{in} \PY{n}{a2}\PY{p}{:}
             \PY{n+nb}{print}\PY{p}{(}\PY{n}{i}\PY{p}{)}
\end{Verbatim}

\begin{Verbatim}[commandchars=\\\{\}]
{\color{outcolor}Out[{\color{outcolor}24}]:} <generator object <genexpr> at 0x000001903CFEC840>
\end{Verbatim}
            
    \begin{Verbatim}[commandchars=\\\{\}]
0
2
4
6
8
10

    \end{Verbatim}

    \hypertarget{ux63a8ux5bfcux5f0f}{%
\subsection{推导式}\label{ux63a8ux5bfcux5f0f}}

可以生成list、dict和set。

\begin{itemize}
\tightlist
\item
  代码简洁
\item
  可读性强
\end{itemize}

{[}表达式 for 变量 in 可迭代对象 if 逻辑条件{]}

\{键值表达式:元素表达式 for 变量 in 可迭代对象 if 逻辑条件\}

\{元素表达式 for 变量 in 可迭代对象 if 逻辑条件\}

    \begin{Verbatim}[commandchars=\\\{\}]
{\color{incolor}In [{\color{incolor}25}]:} \PY{n}{a}\PY{o}{=}\PY{p}{[}\PY{l+m+mi}{0}\PY{p}{,}\PY{l+m+mi}{1}\PY{p}{,}\PY{l+m+mi}{2}\PY{p}{,}\PY{l+m+mi}{3}\PY{p}{,}\PY{l+m+mi}{4}\PY{p}{,}\PY{l+m+mi}{5}\PY{p}{]}
         \PY{p}{[}\PY{n}{i}\PY{o}{+}\PY{l+m+mi}{100} \PY{k}{for} \PY{n}{i} \PY{o+ow}{in} \PY{n}{a} \PY{k}{if} \PY{n}{i}\PY{o}{\PYZpc{}}\PY{k}{2}==0]
\end{Verbatim}

\begin{Verbatim}[commandchars=\\\{\}]
{\color{outcolor}Out[{\color{outcolor}25}]:} [100, 102, 104]
\end{Verbatim}
            
    \begin{Verbatim}[commandchars=\\\{\}]
{\color{incolor}In [{\color{incolor}26}]:} \PY{p}{\PYZob{}}\PY{n}{k}\PY{p}{:}\PY{n}{v} \PY{k}{for} \PY{n}{k}\PY{p}{,}\PY{n}{v} \PY{o+ow}{in} \PY{n+nb}{zip}\PY{p}{(}\PY{l+s+s1}{\PYZsq{}}\PY{l+s+s1}{推导式}\PY{l+s+s1}{\PYZsq{}}\PY{p}{,}\PY{n+nb}{range}\PY{p}{(}\PY{l+m+mi}{3}\PY{p}{)}\PY{p}{)}\PY{p}{\PYZcb{}}
\end{Verbatim}

\begin{Verbatim}[commandchars=\\\{\}]
{\color{outcolor}Out[{\color{outcolor}26}]:} \{'推': 0, '导': 1, '式': 2\}
\end{Verbatim}
            
    \hypertarget{ux51fdux6570ux5f0fux7f16ux7a0b}{%
\section{函数式编程}\label{ux51fdux6570ux5f0fux7f16ux7a0b}}

    \begin{Verbatim}[commandchars=\\\{\}]
{\color{incolor}In [{\color{incolor}27}]:} \PY{k}{def} \PY{n+nf}{mysum}\PY{p}{(}\PY{n}{num1}\PY{p}{,}\PY{n}{num2}\PY{p}{)}\PY{p}{:}
             \PY{n}{num}\PY{o}{=}\PY{n}{num1}\PY{o}{+}\PY{n}{num2}
             \PY{k}{return} \PY{n}{num}
         \PY{n}{mysum}\PY{p}{(}\PY{l+m+mi}{1}\PY{p}{,}\PY{l+m+mi}{2}\PY{p}{)}
\end{Verbatim}

\begin{Verbatim}[commandchars=\\\{\}]
{\color{outcolor}Out[{\color{outcolor}27}]:} 3
\end{Verbatim}
            
    通过上面的学习,可以知道通过return语句返回值。

不带参数值的 return 语句返回 None。

函数可以返回多个值。
该函数其实只返回了一个对象,也就是一个元组,最后该元组会被拆包到各个结果变量中。

    \begin{Verbatim}[commandchars=\\\{\}]
{\color{incolor}In [{\color{incolor}28}]:} \PY{k}{def} \PY{n+nf}{f}\PY{p}{(}\PY{p}{)}\PY{p}{:}
             \PY{n}{a} \PY{o}{=} \PY{l+m+mi}{5}
             \PY{n}{b} \PY{o}{=} \PY{l+m+mi}{6}
             \PY{n}{c} \PY{o}{=} \PY{l+m+mi}{7}
             \PY{k}{return} \PY{n}{a}\PY{p}{,} \PY{n}{b}\PY{p}{,} \PY{n}{c}
         
         \PY{n}{a}\PY{p}{,} \PY{n}{b}\PY{p}{,} \PY{n}{c} \PY{o}{=} \PY{n}{f}\PY{p}{(}\PY{p}{)}
         \PY{n}{a}
         \PY{n}{b}
         \PY{n}{c}
\end{Verbatim}

\begin{Verbatim}[commandchars=\\\{\}]
{\color{outcolor}Out[{\color{outcolor}28}]:} 5
\end{Verbatim}
            
\begin{Verbatim}[commandchars=\\\{\}]
{\color{outcolor}Out[{\color{outcolor}28}]:} 6
\end{Verbatim}
            
\begin{Verbatim}[commandchars=\\\{\}]
{\color{outcolor}Out[{\color{outcolor}28}]:} 7
\end{Verbatim}
            
    默认参数

    \begin{Verbatim}[commandchars=\\\{\}]
{\color{incolor}In [{\color{incolor}29}]:} \PY{k}{def} \PY{n+nf}{mysum}\PY{p}{(}\PY{n}{num1}\PY{o}{=}\PY{l+m+mi}{0}\PY{p}{,}\PY{n}{num2}\PY{o}{=}\PY{l+m+mi}{0}\PY{p}{)}\PY{p}{:}
             \PY{n}{num}\PY{o}{=}\PY{n}{num1}\PY{o}{+}\PY{n}{num2}
             \PY{k}{return} \PY{n}{num}
         \PY{n}{mysum}\PY{p}{(}\PY{n}{num2}\PY{o}{=}\PY{l+m+mi}{100}\PY{p}{)}
\end{Verbatim}

\begin{Verbatim}[commandchars=\\\{\}]
{\color{outcolor}Out[{\color{outcolor}29}]:} 100
\end{Verbatim}
            
    匿名函数

    \begin{Verbatim}[commandchars=\\\{\}]
{\color{incolor}In [{\color{incolor}30}]:} \PY{n+nb}{list}\PY{p}{(}\PY{n+nb}{map}\PY{p}{(}\PY{k}{lambda} \PY{n}{x}\PY{p}{:} \PY{n}{x} \PY{o}{*} \PY{n}{x}\PY{p}{,} \PY{p}{[}\PY{l+m+mi}{1}\PY{p}{,} \PY{l+m+mi}{2}\PY{p}{,} \PY{l+m+mi}{3}\PY{p}{,} \PY{l+m+mi}{4}\PY{p}{]}\PY{p}{)}\PY{p}{)}
\end{Verbatim}

\begin{Verbatim}[commandchars=\\\{\}]
{\color{outcolor}Out[{\color{outcolor}30}]:} [1, 4, 9, 16]
\end{Verbatim}
            
    此外还有

\begin{itemize}
\tightlist
\item
  作用域
\item
  闭包
\item
  装饰器
\end{itemize}

    \hypertarget{ux9762ux5411ux5bf9ux8c61ux7f16ux7a0b}{%
\section{面向对象编程}\label{ux9762ux5411ux5bf9ux8c61ux7f16ux7a0b}}

    Python中的所有事物都是以对象形式存在,从简单的数值类型,到复杂的代码模块,都是对象。

同一个类(class)的对象具有相同的属性和方法。

对象实现了属性和方法的封装,是一种数据抽象机制。

面向对象三个特点:

\begin{itemize}
\tightlist
\item
  继承
\item
  多态
\item
  封装性
\end{itemize}

平时的分析场景不需要多少面向对象编程的地方,简单使用即可。属性+方法。

    下面举一个例子,学生类型,和博士生子类型。

    \begin{Verbatim}[commandchars=\\\{\}]
{\color{incolor}In [{\color{incolor}31}]:} \PY{k}{class} \PY{n+nc}{Student}\PY{p}{:}
             \PY{k}{def} \PY{n+nf}{\PYZus{}\PYZus{}init\PYZus{}\PYZus{}}\PY{p}{(}\PY{n+nb+bp}{self}\PY{p}{,} \PY{n+nb}{id}\PY{p}{,} \PY{n}{score}\PY{p}{)}\PY{p}{:}
                 \PY{n+nb+bp}{self}\PY{o}{.}\PY{n}{id} \PY{o}{=} \PY{n+nb}{id}
                 \PY{n+nb+bp}{self}\PY{o}{.}\PY{n}{score} \PY{o}{=} \PY{n}{score}
             \PY{k}{def} \PY{n+nf}{pressure}\PY{p}{(}\PY{n+nb+bp}{self}\PY{p}{)}\PY{p}{:}
                 \PY{k}{if} \PY{n+nb+bp}{self}\PY{o}{.}\PY{n}{score}\PY{o}{\PYZgt{}}\PY{l+m+mi}{80}\PY{p}{:}
                     \PY{n+nb}{print}\PY{p}{(}\PY{l+s+s2}{\PYZdq{}}\PY{l+s+s2}{还不错}\PY{l+s+s2}{\PYZdq{}}\PY{p}{)} 
                 \PY{k}{else}\PY{p}{:}
                     \PY{n+nb}{print}\PY{p}{(}\PY{l+s+s2}{\PYZdq{}}\PY{l+s+s2}{需要加油}\PY{l+s+s2}{\PYZdq{}}\PY{p}{)}
         
         \PY{k}{class} \PY{n+nc}{Phd}\PY{p}{(}\PY{n}{Student}\PY{p}{)}\PY{p}{:}
             \PY{k}{def} \PY{n+nf}{\PYZus{}\PYZus{}init\PYZus{}\PYZus{}}\PY{p}{(}\PY{n+nb+bp}{self}\PY{p}{,} \PY{n+nb}{id}\PY{p}{,} \PY{n}{score}\PY{p}{,} \PY{n}{paper}\PY{p}{)}\PY{p}{:}
                 \PY{n}{Student}\PY{o}{.}\PY{n+nf+fm}{\PYZus{}\PYZus{}init\PYZus{}\PYZus{}}\PY{p}{(}\PY{n+nb+bp}{self}\PY{p}{,} \PY{n+nb}{id}\PY{p}{,} \PY{n}{score}\PY{p}{)}  
                 \PY{c+c1}{\PYZsh{} 或super().\PYZus{}\PYZus{}init\PYZus{}\PYZus{}(id, score)}
                 \PY{n+nb+bp}{self}\PY{o}{.}\PY{n}{paper} \PY{o}{=} \PY{n}{paper}
                 
             \PY{k}{def} \PY{n+nf}{pressure}\PY{p}{(}\PY{n+nb+bp}{self}\PY{p}{)}\PY{p}{:}
                 \PY{k}{if} \PY{n+nb+bp}{self}\PY{o}{.}\PY{n}{score}\PY{o}{\PYZgt{}}\PY{l+m+mi}{60}\PY{p}{:}
                     \PY{n+nb}{print}\PY{p}{(}\PY{l+s+s2}{\PYZdq{}}\PY{l+s+s2}{还不错}\PY{l+s+s2}{\PYZdq{}}\PY{p}{)} 
                 \PY{k}{else}\PY{p}{:}
                     \PY{n+nb}{print}\PY{p}{(}\PY{l+s+s2}{\PYZdq{}}\PY{l+s+s2}{需要加油}\PY{l+s+s2}{\PYZdq{}}\PY{p}{)}
                 
             \PY{k}{def} \PY{n+nf}{pressure\PYZus{}paper}\PY{p}{(}\PY{n+nb+bp}{self}\PY{p}{,}\PY{n}{number}\PY{p}{)}\PY{p}{:}
                 \PY{k}{if} \PY{n+nb+bp}{self}\PY{o}{.}\PY{n}{paper}\PY{o}{==}\PY{l+s+s2}{\PYZdq{}}\PY{l+s+s2}{A}\PY{l+s+s2}{\PYZdq{}} \PY{o+ow}{and} \PY{n}{number}\PY{o}{\PYZgt{}}\PY{o}{=}\PY{l+m+mi}{1}\PY{p}{:}
                     \PY{k}{return} \PY{n}{f}\PY{l+s+s2}{\PYZdq{}}\PY{l+s+si}{\PYZob{}self.id\PYZcb{}}\PY{l+s+s2}{发了}\PY{l+s+si}{\PYZob{}self.paper\PYZcb{}}\PY{l+s+s2}{刊共计}\PY{l+s+si}{\PYZob{}number\PYZcb{}}\PY{l+s+s2}{篇,可以参加预答辩了}\PY{l+s+s2}{\PYZdq{}}
                 \PY{k}{elif} \PY{n+nb+bp}{self}\PY{o}{.}\PY{n}{paper}\PY{o}{==}\PY{l+s+s2}{\PYZdq{}}\PY{l+s+s2}{B}\PY{l+s+s2}{\PYZdq{}} \PY{o+ow}{and} \PY{n}{number}\PY{o}{\PYZgt{}}\PY{o}{=}\PY{l+m+mi}{2}\PY{p}{:}
                     \PY{k}{return} \PY{n}{f}\PY{l+s+s2}{\PYZdq{}}\PY{l+s+si}{\PYZob{}self.id\PYZcb{}}\PY{l+s+s2}{发了}\PY{l+s+si}{\PYZob{}self.paper\PYZcb{}}\PY{l+s+s2}{刊共计}\PY{l+s+si}{\PYZob{}number\PYZcb{}}\PY{l+s+s2}{篇,可以参加预答辩了}\PY{l+s+s2}{\PYZdq{}}
                 \PY{k}{else}\PY{p}{:}
                     \PY{k}{return} \PY{n}{f}\PY{l+s+s2}{\PYZdq{}}\PY{l+s+si}{\PYZob{}self.id\PYZcb{}}\PY{l+s+s2}{发了}\PY{l+s+si}{\PYZob{}self.paper\PYZcb{}}\PY{l+s+s2}{刊共计}\PY{l+s+si}{\PYZob{}number\PYZcb{}}\PY{l+s+s2}{篇,尚未达到预答辩条件}\PY{l+s+s2}{\PYZdq{}}
\end{Verbatim}

    \begin{Verbatim}[commandchars=\\\{\}]
{\color{incolor}In [{\color{incolor}32}]:} \PY{n}{zhang3}\PY{o}{=}\PY{n}{Student}\PY{p}{(}\PY{l+s+s2}{\PYZdq{}}\PY{l+s+s2}{2018001}\PY{l+s+s2}{\PYZdq{}}\PY{p}{,}\PY{l+m+mi}{81}\PY{p}{)}
         \PY{n}{zhang3}\PY{o}{.}\PY{n}{id}
         \PY{n}{zhang3}\PY{o}{.}\PY{n}{score}
         \PY{n}{zhang3}\PY{o}{.}\PY{n}{pressure}\PY{p}{(}\PY{p}{)}
\end{Verbatim}

\begin{Verbatim}[commandchars=\\\{\}]
{\color{outcolor}Out[{\color{outcolor}32}]:} '2018001'
\end{Verbatim}
            
\begin{Verbatim}[commandchars=\\\{\}]
{\color{outcolor}Out[{\color{outcolor}32}]:} 81
\end{Verbatim}
            
    \begin{Verbatim}[commandchars=\\\{\}]
还不错

    \end{Verbatim}

    \begin{Verbatim}[commandchars=\\\{\}]
{\color{incolor}In [{\color{incolor}33}]:} \PY{n}{li4}\PY{o}{=}\PY{n}{Phd}\PY{p}{(}\PY{l+s+s2}{\PYZdq{}}\PY{l+s+s2}{2018002}\PY{l+s+s2}{\PYZdq{}}\PY{p}{,}\PY{l+m+mi}{61}\PY{p}{,}\PY{l+s+s2}{\PYZdq{}}\PY{l+s+s2}{B}\PY{l+s+s2}{\PYZdq{}}\PY{p}{)}
         \PY{n}{li4}\PY{o}{.}\PY{n}{id}
         \PY{n}{li4}\PY{o}{.}\PY{n}{score}
         \PY{n}{li4}\PY{o}{.}\PY{n}{paper}
         \PY{n}{li4}\PY{o}{.}\PY{n}{pressure}\PY{p}{(}\PY{p}{)}
         \PY{n}{li4}\PY{o}{.}\PY{n}{pressure\PYZus{}paper}\PY{p}{(}\PY{l+m+mi}{2}\PY{p}{)}
\end{Verbatim}

\begin{Verbatim}[commandchars=\\\{\}]
{\color{outcolor}Out[{\color{outcolor}33}]:} '2018002'
\end{Verbatim}
            
\begin{Verbatim}[commandchars=\\\{\}]
{\color{outcolor}Out[{\color{outcolor}33}]:} 61
\end{Verbatim}
            
\begin{Verbatim}[commandchars=\\\{\}]
{\color{outcolor}Out[{\color{outcolor}33}]:} 'B'
\end{Verbatim}
            
    \begin{Verbatim}[commandchars=\\\{\}]
还不错

    \end{Verbatim}

\begin{Verbatim}[commandchars=\\\{\}]
{\color{outcolor}Out[{\color{outcolor}33}]:} '2018002发了B刊共计2篇,可以参加预答辩了'
\end{Verbatim}
            
    \hypertarget{ux6570ux636eux8bfbux5199}{%
\section{数据读写}\label{ux6570ux636eux8bfbux5199}}

    \hypertarget{ux7b80ux5355ux8f93ux5165ux548cux8f93ux51fa}{%
\subsection{简单输入和输出}\label{ux7b80ux5355ux8f93ux5165ux548cux8f93ux51fa}}

    \begin{Verbatim}[commandchars=\\\{\}]
{\color{incolor}In [{\color{incolor}34}]:} \PY{n}{use\PYZus{}name}\PY{o}{=}\PY{n+nb}{input}\PY{p}{(}\PY{l+s+s2}{\PYZdq{}}\PY{l+s+s2}{请输入你的名字}\PY{l+s+s2}{\PYZdq{}}\PY{p}{)}
\end{Verbatim}

    \begin{Verbatim}[commandchars=\\\{\}]
请输入你的名字 cheng jun

    \end{Verbatim}

    \begin{Verbatim}[commandchars=\\\{\}]
{\color{incolor}In [{\color{incolor}35}]:} \PY{n}{use\PYZus{}name}
\end{Verbatim}

\begin{Verbatim}[commandchars=\\\{\}]
{\color{outcolor}Out[{\color{outcolor}35}]:} 'cheng jun'
\end{Verbatim}
            
    print(v1,v2,v3)打印各个变量

    \begin{Verbatim}[commandchars=\\\{\}]
{\color{incolor}In [{\color{incolor}36}]:} \PY{n+nb}{print}\PY{p}{(}\PY{l+s+s2}{\PYZdq{}}\PY{l+s+s2}{打印}\PY{l+s+s2}{\PYZdq{}}\PY{p}{,}\PY{l+s+s2}{\PYZdq{}}\PY{l+s+s2}{各个变量}\PY{l+s+s2}{\PYZdq{}}\PY{p}{)}
\end{Verbatim}

    \begin{Verbatim}[commandchars=\\\{\}]
打印 各个变量

    \end{Verbatim}

    \hypertarget{ux57faux7840io}{%
\subsection{基础IO}\label{ux57faux7840io}}

\begin{itemize}
\tightlist
\item
  open
\item
  read或者write
\item
  close
\end{itemize}

由于文件读写时都有可能产生IOError,所以还要借助try函数。

\begin{Shaded}
\begin{Highlighting}[]
\ControlFlowTok{try}\NormalTok{:}
\NormalTok{    f }\OperatorTok{=} \BuiltInTok{open}\NormalTok{(}\StringTok{'Python-Data-Science/file.txt'}\NormalTok{, }\StringTok{'r'}\NormalTok{)}
    \BuiltInTok{print}\NormalTok{(f.read())}
\ControlFlowTok{finally}\NormalTok{:}
    \ControlFlowTok{if}\NormalTok{ f:}
\NormalTok{        f.close()}
\end{Highlighting}
\end{Shaded}

这句话就可以保证无论是否出错都能正确地关闭文件。

这样太麻烦。所以就引出with语句。

\begin{Shaded}
\begin{Highlighting}[]
\ControlFlowTok{with} \BuiltInTok{open}\NormalTok{(}\StringTok{'Python-Data-Science/file.txt'}\NormalTok{, }\StringTok{'r'}\NormalTok{, encoding}\OperatorTok{=}\StringTok{'gbk'}\NormalTok{) }\ImportTok{as}\NormalTok{ f:}
    \BuiltInTok{print}\NormalTok{(f.read())}
\end{Highlighting}
\end{Shaded}

写文件和读文件是一样的,唯一区别是调用open()函数时的参数不同:

\begin{itemize}
\item
  传入标识符\texttt{r}或者\texttt{rb}表示写文本文件或写二进制文件
\item
  传入标识符\texttt{w}或者\texttt{wb}表示写文本文件或写二进制文件
\item
  在open()函数传入encoding参数,将字符串自动转换成指定编码
\item
  以'w'模式写入文件时,如果文件已存在,会直接覆盖已有文件
\item
  以'a'模式写入文件时,以追加(append)模式写入
\end{itemize}

    \hypertarget{csv}{%
\subsection{csv}\label{csv}}

这个需要csv包,或者是pandas

    \hypertarget{ux4e8cux8fdbux5236ux6587ux4ef6}{%
\subsection{二进制文件}\label{ux4e8cux8fdbux5236ux6587ux4ef6}}

要读取二进制文件,比如图片、视频等等,用\texttt{rb}模式打开文件即可,写入二进制文件,用\texttt{wb}即可。

\begin{Shaded}
\begin{Highlighting}[]
\NormalTok{f }\OperatorTok{=} \BuiltInTok{open}\NormalTok{(}\StringTok{'Python-Data-Science/image/lambda.png'}\NormalTok{, }\StringTok{'rb'}\NormalTok{)}
\NormalTok{f.read()[}\DecValTok{0}\NormalTok{:}\DecValTok{10}\NormalTok{]}
\end{Highlighting}
\end{Shaded}

    \hypertarget{ux6587ux672cux5904ux7406}{%
\section{文本处理}\label{ux6587ux672cux5904ux7406}}

常见四把斧子。

\begin{itemize}
\tightlist
\item
  Python的字符串属性函数
\item
  Python的string模块
\item
  re
\item
  其他
\end{itemize}

    \hypertarget{ux5f02ux5e38ux5904ux7406}{%
\section{异常处理}\label{ux5f02ux5e38ux5904ux7406}}

Python在builtins模块内置了常见的异常,无需特别导入,直接就可使用。

    \begin{Verbatim}[commandchars=\\\{\}]
{\color{incolor}In [{\color{incolor}37}]:} \PY{k}{try}\PY{p}{:}
             \PY{n}{a} \PY{o}{=} \PY{l+m+mi}{1} \PY{o}{/} \PY{l+m+mi}{0}
         \PY{k}{except} \PY{n+ne}{ZeroDivisionError} \PY{k}{as} \PY{n}{e}\PY{p}{:}
             \PY{n+nb}{print}\PY{p}{(}\PY{n}{e}\PY{p}{)}
         \PY{k}{else}\PY{p}{:}
             \PY{n+nb}{print}\PY{p}{(}\PY{l+s+s2}{\PYZdq{}}\PY{l+s+s2}{没有错误}\PY{l+s+s2}{\PYZdq{}}\PY{p}{)}
         \PY{k}{finally}\PY{p}{:}
             \PY{n+nb}{print}\PY{p}{(}\PY{l+s+s2}{\PYZdq{}}\PY{l+s+s2}{运行结束。本语句一定会运行。}\PY{l+s+s2}{\PYZdq{}}\PY{p}{)}
\end{Verbatim}

    \begin{Verbatim}[commandchars=\\\{\}]
division by zero
运行结束。本语句一定会运行。

    \end{Verbatim}

    通用异常:Exception

在Python的异常中,有一个通用异常:Exception,它可以捕获任意异常。

    \begin{Verbatim}[commandchars=\\\{\}]
{\color{incolor}In [{\color{incolor}38}]:} \PY{k}{try}\PY{p}{:}
             \PY{n+nb}{print}\PY{p}{(}\PY{l+m+mi}{1}\PY{o}{/}\PY{l+m+mi}{0}\PY{p}{)}
         \PY{k}{except} \PY{n+ne}{Exception} \PY{k}{as} \PY{n}{e}\PY{p}{:}
             \PY{n+nb}{print}\PY{p}{(}\PY{l+s+s2}{\PYZdq{}}\PY{l+s+s2}{我去,又出错了。}\PY{l+s+se}{\PYZbs{}n}\PY{l+s+s2}{错误提示信息:}\PY{l+s+s2}{\PYZdq{}}\PY{p}{,} \PY{n}{e}\PY{p}{,} \PY{l+s+s2}{\PYZdq{}}\PY{l+s+s2}{。}\PY{l+s+s2}{\PYZdq{}}\PY{p}{)}
\end{Verbatim}

    \begin{Verbatim}[commandchars=\\\{\}]
我去,又出错了。
错误提示信息: division by zero 。

    \end{Verbatim}

    \hypertarget{ux8c03ux8bd5ux4e0eux6d4bux8bd5}{%
\section{调试与测试}\label{ux8c03ux8bd5ux4e0eux6d4bux8bd5}}

    \hypertarget{ux8c03ux8bd5}{%
\subsection{调试}\label{ux8c03ux8bd5}}

\begin{itemize}
\tightlist
\item
  print
\item
  assert
\item
  logging
\item
  pdb
\item
  IDE
\end{itemize}

    asert会对后面的表达式进行判断,如果表达式为True,那就什么都不做,程序接着往下走;如果False,那么就会弹出异常。

    \begin{Verbatim}[commandchars=\\\{\}]
{\color{incolor}In [{\color{incolor}39}]:} \PY{k}{def} \PY{n+nf}{myfun}\PY{p}{(}\PY{n}{s}\PY{p}{)}\PY{p}{:}
             \PY{k}{assert} \PY{n+nb}{type}\PY{p}{(}\PY{n}{s}\PY{p}{)}\PY{o}{==}\PY{n+nb}{int}
             \PY{k}{return} \PY{n}{s}\PY{o}{/}\PY{l+m+mi}{2}
\end{Verbatim}

    \begin{Verbatim}[commandchars=\\\{\}]
{\color{incolor}In [{\color{incolor}40}]:} \PY{n}{myfun}\PY{p}{(}\PY{l+m+mi}{18}\PY{p}{)}  \PY{c+c1}{\PYZsh{} 正常运行}
\end{Verbatim}

\begin{Verbatim}[commandchars=\\\{\}]
{\color{outcolor}Out[{\color{outcolor}40}]:} 9.0
\end{Verbatim}
            
    \begin{Verbatim}[commandchars=\\\{\}]
{\color{incolor}In [{\color{incolor}41}]:} \PY{n}{myfun}\PY{p}{(}\PY{l+m+mf}{18.0}\PY{p}{)}  \PY{c+c1}{\PYZsh{} 触发错误}
\end{Verbatim}

    \begin{Verbatim}[commandchars=\\\{\}]

        ---------------------------------------------------------------------------

        AssertionError                            Traceback (most recent call last)

        <ipython-input-41-28fb6113b9bb> in <module>()
    ----> 1 myfun(18.0)  \# 触发错误
    

        <ipython-input-39-1715a7599229> in myfun(s)
          1 def myfun(s):
    ----> 2     assert type(s)==int
          3     return s/2
    

        AssertionError: 

    \end{Verbatim}

    \hypertarget{ux6027ux80fdux6d4bux8bd5}{%
\subsection{性能测试}\label{ux6027ux80fdux6d4bux8bd5}}

Python提供timeit来测试性能。

profile 和 pstats 模块提供了针对更大代码块的性能测试。

    \begin{Verbatim}[commandchars=\\\{\}]
{\color{incolor}In [{\color{incolor}42}]:} \PY{k+kn}{from} \PY{n+nn}{timeit} \PY{k}{import} \PY{n}{timeit}
         
         \PY{k}{def} \PY{n+nf}{myfun}\PY{p}{(}\PY{p}{)}\PY{p}{:}
             \PY{p}{[}\PY{n}{i} \PY{k}{for} \PY{n}{i} \PY{o+ow}{in} \PY{n+nb}{range}\PY{p}{(}\PY{l+m+mi}{1000000}\PY{p}{)} \PY{k}{if} \PY{n}{i}\PY{o}{\PYZpc{}}\PY{k}{3}==0]
\end{Verbatim}

    timeit(函数名\_字符串,运行环境\_字符串,number=运行次数)

    \begin{Verbatim}[commandchars=\\\{\}]
{\color{incolor}In [{\color{incolor}44}]:} \PY{n}{timeit}\PY{p}{(}\PY{l+s+s1}{\PYZsq{}}\PY{l+s+s1}{myfun()}\PY{l+s+s1}{\PYZsq{}}\PY{p}{,} \PY{l+s+s1}{\PYZsq{}}\PY{l+s+s1}{from \PYZus{}\PYZus{}main\PYZus{}\PYZus{} import myfun}\PY{l+s+s1}{\PYZsq{}}\PY{p}{,} \PY{n}{number}\PY{o}{=}\PY{l+m+mi}{2}\PY{p}{)}
\end{Verbatim}

\begin{Verbatim}[commandchars=\\\{\}]
{\color{outcolor}Out[{\color{outcolor}44}]:} 0.13829108199999496
\end{Verbatim}
            
    \begin{Verbatim}[commandchars=\\\{\}]
{\color{incolor}In [{\color{incolor}45}]:} \PY{n}{timeit}\PY{p}{(}\PY{l+s+s1}{\PYZsq{}}\PY{l+s+s1}{[i for i in range(1000000) if i}\PY{l+s+s1}{\PYZpc{}}\PY{l+s+s1}{3==0]}\PY{l+s+s1}{\PYZsq{}}\PY{p}{,} \PY{n}{number}\PY{o}{=}\PY{l+m+mi}{2}\PY{p}{)}
\end{Verbatim}

\begin{Verbatim}[commandchars=\\\{\}]
{\color{outcolor}Out[{\color{outcolor}45}]:} 0.13907842400001869
\end{Verbatim}
            
    \begin{Verbatim}[commandchars=\\\{\}]
{\color{incolor}In [{\color{incolor}46}]:} \PY{o}{\PYZpc{}}\PY{k}{time} myfun()
\end{Verbatim}

    \begin{Verbatim}[commandchars=\\\{\}]
Wall time: 62.5 ms

    \end{Verbatim}

    此外还可以使用repeat方法。

\begin{Shaded}
\begin{Highlighting}[]
\ImportTok{from}\NormalTok{ timeit }\ImportTok{import}\NormalTok{ repeat}
\ControlFlowTok{pass}
\NormalTok{repeat(}\StringTok{'func()'}\NormalTok{, }\StringTok{'from __main__ import func'}\NormalTok{, number}\OperatorTok{=}\DecValTok{100}\NormalTok{, repeat}\OperatorTok{=}\DecValTok{5}\NormalTok{)}
\end{Highlighting}
\end{Shaded}

    \hypertarget{ux5355ux5143ux6d4bux8bd5}{%
\subsection{单元测试}\label{ux5355ux5143ux6d4bux8bd5}}

做探索性数据分析的时候很少做单元测试。

跳过。

    \hypertarget{ux6a21ux5757ux4e0eux5305}{%
\section{模块与包}\label{ux6a21ux5757ux4e0eux5305}}

    \hypertarget{ux6a21ux5757}{%
\subsection{模块}\label{ux6a21ux5757}}

    \hypertarget{ux5bfcux5165}{%
\subsubsection{导入}\label{ux5bfcux5165}}

模块主要是以下导入方式。

\begin{Shaded}
\begin{Highlighting}[]
\ImportTok{import}\NormalTok{ xx.xx}

\ImportTok{from}\NormalTok{ xx.xx }\ImportTok{import}\NormalTok{ xx}

\CommentTok{# 给导入的对象重命名}
\ImportTok{from}\NormalTok{ xx.xx }\ImportTok{import}\NormalTok{ xx }\ImportTok{as}\NormalTok{ newname}

\CommentTok{# 将对象内的所有内容全部导入,容易发生命名冲突,慎用}
\ImportTok{from}\NormalTok{ xx.xx }\ImportTok{import} \OperatorTok{*}
\end{Highlighting}
\end{Shaded}

在导入的时候,需要提供导入对象的绝对路径,也就是``最顶层的包名.次一级包名.模块名.类名.函数名''。

对于xx.xx,想导入到哪个级别,取决于自己需要。

标准库,直接import 模块名,例如\texttt{import\ os}。

    \hypertarget{ux4e3bux6a21ux5757ux548cux975eux4e3bux6a21ux5757}{%
\subsubsection{主模块和非主模块}\label{ux4e3bux6a21ux5757ux548cux975eux4e3bux6a21ux5757}}

如果一个模块被直接使用,而没有被别人调用,称这个模块为主模块。
如果一个模块被别人调用,称这个模块为非主模块。

\_\_name\_\_属性值是一个变量,且这个变量是系统给出的。利用这个变量可以判断一个模块是否是主模块。如果一个属性的值是
\textbf{main} ,那么就说明这个模块是主模块。

    \begin{Shaded}
\begin{Highlighting}[]
\ControlFlowTok{if} \VariableTok{__name__} \OperatorTok{==} \StringTok{'__main__'}\NormalTok{:}
\NormalTok{    main()}
\end{Highlighting}
\end{Shaded}

上面的代码很常见。

在同一目录建立以下代码。

\begin{Shaded}
\begin{Highlighting}[]
\KeywordTok{def}\NormalTok{ myfun():}
    \BuiltInTok{print}\NormalTok{(}\VariableTok{__name__}\NormalTok{)}

\ControlFlowTok{if} \VariableTok{__name__}\OperatorTok{==}\StringTok{'__main__'}\NormalTok{:}
\NormalTok{    myfun()}
\end{Highlighting}
\end{Shaded}

上述代码运行结果:

\begin{Shaded}
\begin{Highlighting}[]
\NormalTok{[Running] python }\StringTok{"f:}\CharTok{\textbackslash{}t}\StringTok{estmain.py"}
\NormalTok{__main__}
\end{Highlighting}
\end{Shaded}

\begin{Shaded}
\begin{Highlighting}[]
\ImportTok{import}\NormalTok{ testmain}

\NormalTok{testmain.myfun()}
\end{Highlighting}
\end{Shaded}

上述代码运行结果:

\begin{Shaded}
\begin{Highlighting}[]
\NormalTok{[Running] python }\StringTok{"test.py"}
\NormalTok{testmain}
\end{Highlighting}
\end{Shaded}

testmain是包的名字testmain。

\begin{itemize}
\item
  \texttt{if\ \_\_name\_\_\ ==\ \textquotesingle{}\_\_main\_\_\textquotesingle{}}自己调用时\texttt{\_\_name\_\_}为\texttt{\textquotesingle{}\_\_main\_\_\textquotesingle{}},条件成立,执行if语句中函数
\item
  \texttt{if\ \_\_name\_\_\ ==\ \textquotesingle{}\_\_main\_\_\textquotesingle{}}从别的文件调用时\texttt{\_\_name\_\_}为'调用的文件名',if条件不成立,则不执行。
\end{itemize}

    \hypertarget{ux5305}{%
\subsection{包}\label{ux5305}}

包是用以组织模块的另一种层次结构。

包是使用包含文件\texttt{\_\_init\_\_.py}的目录实现的。

    \hypertarget{pipux8f6fux4ef6ux5305ux7ba1ux7406}{%
\subsection{pip软件包管理}\label{pipux8f6fux4ef6ux5305ux7ba1ux7406}}

\hypertarget{ux5b89ux88c5}{%
\subsubsection{安装}\label{ux5b89ux88c5}}

\begin{Shaded}
\begin{Highlighting}[]
\NormalTok{pip install packageName 下载并安装最新的版本}
\NormalTok{pip install packageName}\OperatorTok{==}\DecValTok{1}\NormalTok{.}\FloatTok{0.0}\NormalTok{下载并安装指定版本}
\NormalTok{pip install packageName}\OperatorTok{>=}\DecValTok{1}\NormalTok{.}\FloatTok{0.0}\NormalTok{ 下载并安装至少某个版本以上的版本的包}
\NormalTok{pip install url }\CommentTok{#从指定网址资源安装}
\NormalTok{pip install path }\CommentTok{#指定本地位置安装}
\NormalTok{pip install }\OperatorTok{--}\NormalTok{find}\OperatorTok{-}\NormalTok{links}\OperatorTok{=}\NormalTok{url 从指定url下载安装}
\NormalTok{pip install }\OperatorTok{--}\NormalTok{find}\OperatorTok{-}\NormalTok{links}\OperatorTok{=}\NormalTok{path 从指定path下载安装}
\NormalTok{pip install }\OperatorTok{--}\NormalTok{upgrade packageName 更新一个已经安装过的过期模块}
\end{Highlighting}
\end{Shaded}

同时安装多个包:python -m pip install --user numpy scipy matplotlib
ipython jupyter pandas sympy nose

提示安转时缺失包:python --m pip install qtconsole

在Win10
的ps界面上使用pip需要管理者权限,或者是在命令后面加上\texttt{-\/-user}。

\hypertarget{ux5378ux8f7d}{%
\subsubsection{卸载}\label{ux5378ux8f7d}}

\begin{verbatim}
pip uninstall <packageName>
\end{verbatim}

\hypertarget{ux67e5ux770bux6a21ux5757ux4fe1ux606f}{%
\subsubsection{查看模块信息}\label{ux67e5ux770bux6a21ux5757ux4fe1ux606f}}

\begin{verbatim}
pip show <packageName>
\end{verbatim}

查看pip管理了哪些模块

\begin{verbatim}
pip list -o  列示过期的包
pip list -u  列示可以更新到最新包的信息
\end{verbatim}

pip升级自己

\begin{verbatim}
python -m pip install --upgrade pip
\end{verbatim}

查看pip版本:

\begin{Shaded}
\begin{Highlighting}[]
\FunctionTok{PS}\NormalTok{ C:\textbackslash{}Users\textbackslash{}cheng> pip -V}
\NormalTok{pip 18.}\FunctionTok{0}
\end{Highlighting}
\end{Shaded}

此外,pip安装在本地有比较大的缓存文件,可以手动检测并删除。

除了匹配管理包,还可以使用conda。

\hypertarget{pypiux955cux50cf}{%
\subsection{pypi镜像}\label{pypiux955cux50cf}}

https://mirrors.tuna.tsinghua.edu.cn/help/pypi/

pypi 镜像每 5 分钟同步一次。

临时使用

\begin{verbatim}
pip install -i https://pypi.tuna.tsinghua.edu.cn/simple some-package
\end{verbatim}

长久使用

找个地方建立pip.ini文件,如\texttt{C:\textbackslash{}Users\textbackslash{}Cheng\textbackslash{}pip\textbackslash{}pip.ini},文件内容如下:

\begin{verbatim}
[global]
index-url = https://pypi.tuna.tsinghua.edu.cn/simple
\end{verbatim}

然后把\texttt{C:\textbackslash{}Users\textbackslash{}Cheng\textbackslash{}pip\textbackslash{}pip.ini}加入到系统环境变量的\texttt{Path}中去。

    \hypertarget{ux6742ux4e03ux6742ux516b}{%
\section{杂七杂八}\label{ux6742ux4e03ux6742ux516b}}

    \hypertarget{ux4e09ux4e2aux5e38ux7528ux7684ux51fdux6570}{%
\subsection{三个常用的函数}\label{ux4e09ux4e2aux5e38ux7528ux7684ux51fdux6570}}

\begin{itemize}
\tightlist
\item
  help
\item
  type
\item
  dir
\end{itemize}

    \begin{Verbatim}[commandchars=\\\{\}]
{\color{incolor}In [{\color{incolor}47}]:} \PY{n}{help}\PY{p}{(}\PY{l+s+s2}{\PYZdq{}}\PY{l+s+s2}{dir}\PY{l+s+s2}{\PYZdq{}}\PY{p}{)}
\end{Verbatim}

    \begin{Verbatim}[commandchars=\\\{\}]
Help on built-in function dir in module builtins:

dir({\ldots})
    dir([object]) -> list of strings
    
    If called without an argument, return the names in the current scope.
    Else, return an alphabetized list of names comprising (some of) the attributes
    of the given object, and of attributes reachable from it.
    If the object supplies a method named \_\_dir\_\_, it will be used; otherwise
    the default dir() logic is used and returns:
      for a module object: the module's attributes.
      for a class object:  its attributes, and recursively the attributes
        of its bases.
      for any other object: its attributes, its class's attributes, and
        recursively the attributes of its class's base classes.


    \end{Verbatim}

    \begin{Verbatim}[commandchars=\\\{\}]
{\color{incolor}In [{\color{incolor}48}]:} \PY{n+nb}{type}\PY{p}{(}\PY{l+m+mi}{108}\PY{p}{)}
\end{Verbatim}

\begin{Verbatim}[commandchars=\\\{\}]
{\color{outcolor}Out[{\color{outcolor}48}]:} int
\end{Verbatim}
            
    \begin{Verbatim}[commandchars=\\\{\}]
{\color{incolor}In [{\color{incolor}49}]:} \PY{n}{a}\PY{o}{=}\PY{p}{[}\PY{l+m+mi}{1}\PY{p}{,}\PY{l+m+mi}{2}\PY{p}{,}\PY{l+m+mi}{3}\PY{p}{]}
         \PY{n+nb}{dir}\PY{p}{(}\PY{n}{a}\PY{p}{)}\PY{p}{[}\PY{o}{\PYZhy{}}\PY{l+m+mi}{5}\PY{p}{:}\PY{p}{:}\PY{p}{]}
\end{Verbatim}

\begin{Verbatim}[commandchars=\\\{\}]
{\color{outcolor}Out[{\color{outcolor}49}]:} ['insert', 'pop', 'remove', 'reverse', 'sort']
\end{Verbatim}
            
    \hypertarget{ux7c7bux578bux63d0ux793a}{%
\subsection{类型提示}\label{ux7c7bux578bux63d0ux793a}}

    \begin{Verbatim}[commandchars=\\\{\}]
{\color{incolor}In [{\color{incolor}50}]:} \PY{n}{a}\PY{p}{:} \PY{n+nb}{int} \PY{o}{=}\PY{l+m+mi}{99}
         \PY{n}{b}\PY{p}{:} \PY{n+nb}{bool} \PY{o}{=} \PY{k+kc}{True}
         \PY{n}{f}\PY{p}{:} \PY{n+nb}{float} \PY{o}{=} \PY{l+m+mf}{1.1}
         \PY{n}{s}\PY{p}{:} \PY{n+nb}{str} \PY{o}{=} \PY{l+s+s2}{\PYZdq{}}\PY{l+s+s2}{abc}\PY{l+s+s2}{\PYZdq{}}
\end{Verbatim}

    复杂的类型和注解,需要引入标准库typing。

    \begin{Verbatim}[commandchars=\\\{\}]
{\color{incolor}In [{\color{incolor}51}]:} \PY{k+kn}{import} \PY{n+nn}{typing}
         
         \PY{k}{def} \PY{n+nf}{myfun}\PY{p}{(}\PY{n}{lst}\PY{p}{:} \PY{n}{typing}\PY{o}{.}\PY{n}{List}\PY{p}{[}\PY{n+nb}{int}\PY{p}{]}\PY{p}{)} \PY{o}{\PYZhy{}}\PY{o}{\PYZgt{}} \PY{n}{typing}\PY{o}{.}\PY{n}{List}\PY{p}{[}\PY{n+nb}{int}\PY{p}{]}\PY{p}{:}
             \PY{n}{lst} \PY{o}{=} \PY{p}{[}\PY{n}{i}\PY{o}{*}\PY{l+m+mi}{10} \PY{k}{for} \PY{n}{i} \PY{o+ow}{in} \PY{n}{lst}\PY{p}{]}
             \PY{k}{return} \PY{n}{lst}
         
         \PY{n}{myfun}\PY{p}{(}\PY{p}{[}\PY{l+m+mi}{1}\PY{p}{,}\PY{l+m+mi}{5}\PY{p}{,}\PY{l+m+mi}{8}\PY{p}{]}\PY{p}{)}
\end{Verbatim}

\begin{Verbatim}[commandchars=\\\{\}]
{\color{outcolor}Out[{\color{outcolor}51}]:} [10, 50, 80]
\end{Verbatim}
            
    \hypertarget{ux5173ux952eux5b57}{%
\subsection{关键字}\label{ux5173ux952eux5b57}}

    \begin{Verbatim}[commandchars=\\\{\}]
{\color{incolor}In [{\color{incolor}52}]:} \PY{k+kn}{import} \PY{n+nn}{keyword}
         \PY{n+nb}{len}\PY{p}{(}\PY{n}{keyword}\PY{o}{.}\PY{n}{kwlist}\PY{p}{)}
         \PY{n}{keyword}\PY{o}{.}\PY{n}{kwlist}
\end{Verbatim}

\begin{Verbatim}[commandchars=\\\{\}]
{\color{outcolor}Out[{\color{outcolor}52}]:} 35
\end{Verbatim}
            
\begin{Verbatim}[commandchars=\\\{\}]
{\color{outcolor}Out[{\color{outcolor}52}]:} ['False',
          'None',
          'True',
          'and',
          'as',
          'assert',
          'async',
          'await',
          'break',
          'class',
          'continue',
          'def',
          'del',
          'elif',
          'else',
          'except',
          'finally',
          'for',
          'from',
          'global',
          'if',
          'import',
          'in',
          'is',
          'lambda',
          'nonlocal',
          'not',
          'or',
          'pass',
          'raise',
          'return',
          'try',
          'while',
          'with',
          'yield']
\end{Verbatim}
            
    \hypertarget{ux5185ux7f6eux51fdux6570}{%
\subsection{内置函数}\label{ux5185ux7f6eux51fdux6570}}

在启动Python解释器的时候,内置函数就已经导入到内存当中供我们使用。

    \begin{Verbatim}[commandchars=\\\{\}]
{\color{incolor}In [{\color{incolor}53}]:} \PY{n+nb}{dir}\PY{p}{(}\PY{n}{\PYZus{}\PYZus{}builtins\PYZus{}\PYZus{}}\PY{p}{)}\PY{p}{[}\PY{l+m+mi}{0}\PY{p}{:}\PY{l+m+mi}{3}\PY{p}{]}
\end{Verbatim}

\begin{Verbatim}[commandchars=\\\{\}]
{\color{outcolor}Out[{\color{outcolor}53}]:} ['ArithmeticError', 'AssertionError', 'AttributeError']
\end{Verbatim}
            
    \begin{Verbatim}[commandchars=\\\{\}]
{\color{incolor}In [{\color{incolor}54}]:} \PY{n+nb}{len}\PY{p}{(}\PY{n+nb}{dir}\PY{p}{(}\PY{n}{\PYZus{}\PYZus{}builtins\PYZus{}\PYZus{}}\PY{p}{)}\PY{p}{)}
\end{Verbatim}

\begin{Verbatim}[commandchars=\\\{\}]
{\color{outcolor}Out[{\color{outcolor}54}]:} 154
\end{Verbatim}
            
    \hypertarget{ux5c0fux6574ux6570ux5bf9ux8c61ux6c60}{%
\subsection{小整数对象池}\label{ux5c0fux6574ux6570ux5bf9ux8c61ux6c60}}

Python初始化的时候会自动建立一个小整数对象池,范围是-5到256。

例如123,即使我们在程序里没有创建它,其实在Python后台已经悄悄为我们创建了。

    \begin{Verbatim}[commandchars=\\\{\}]
{\color{incolor}In [{\color{incolor}55}]:} \PY{n}{a}\PY{o}{=}\PY{l+m+mi}{123}
         \PY{n+nb}{id}\PY{p}{(}\PY{n}{a}\PY{p}{)}\PY{o}{==}\PY{n+nb}{id}\PY{p}{(}\PY{l+m+mi}{123}\PY{p}{)}
\end{Verbatim}

\begin{Verbatim}[commandchars=\\\{\}]
{\color{outcolor}Out[{\color{outcolor}55}]:} True
\end{Verbatim}
            
    \begin{Verbatim}[commandchars=\\\{\}]
{\color{incolor}In [{\color{incolor}56}]:} \PY{n}{a}\PY{o}{=}\PY{l+m+mi}{1230}
         \PY{n+nb}{id}\PY{p}{(}\PY{n}{a}\PY{p}{)}\PY{o}{==}\PY{n+nb}{id}\PY{p}{(}\PY{l+m+mi}{1230}\PY{p}{)}
\end{Verbatim}

\begin{Verbatim}[commandchars=\\\{\}]
{\color{outcolor}Out[{\color{outcolor}56}]:} False
\end{Verbatim}
            
    \hypertarget{isux548c}{%
\subsection{is和==}\label{isux548c}}

\begin{itemize}
\item
  is是比较两个引用是否指向了同一个对象(引用比较)。
\item
  ==是比较两个对象是否相等。
\end{itemize}

    \hypertarget{ux672aux5904ux7406ux77e5ux8bc6ux70b9}{%
\section{未处理知识点}\label{ux672aux5904ux7406ux77e5ux8bc6ux70b9}}

\begin{itemize}
\tightlist
\item
  线程
\item
  进程
\end{itemize}

这些知识点,因为自己平时用的少,学得也是一般。跳过。

    \hypertarget{ux540eux8bb0}{%
\section{后记}\label{ux540eux8bb0}}

在刚学习Python就开始记笔记,因为那个时候Visual
Basic和C长久不用,具体知识点基本遗忘。Python是在这样的背景下,因为有现实的需求,而重拾编程,开始学习的新语言。

一开始的笔记繁杂,几经删改,最后发现没有必要做一个cookbook式的笔记。简单,有体系,方便速览。有事就搜索。

后面有需要,并且有空的时候会对这份笔记进行修订和更新。


    
    
    
\end{document}
